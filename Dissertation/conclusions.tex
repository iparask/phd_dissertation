% !TEX root = main.tex

This dissertation investigated the execution of computational campaigns with
data- and compute-intensive workflows on HPC resources. Our contributions to the
field offer to domain scientists a better understanding of the capabilities
needed to efficiently and effectively execute computational campaigns on HPC
resources. Further, we provide the computational tools, methodologies and
experimental understanding needed to design a software manager to plan and
execute computational campaigns on high performance computing (HPC) 
platforms.

\section{Contributions}

%% Chapter 2
Computational campaigns on HPC resources can execute compute- or data-
intensive workflows. Compute-intensive workflows are at the epicenter of
scientific computing that utilizes HPC and are the net producers of an 
immense amount of data. Compute intensive workflows execute either a 
single long running executable or an ensemble of compute-intensive 
tasks~\cite{balasubramanian2018harnessing}. As the need for analyzing
data on HPC resources increases, the efficient and effective execution of 
data-intensive workflows on HPC becomes important. Contrary to 
compute-intensive, data-intensive workflows execute a large number of 
short-running tasks in multiple stages which can be I/O, memory and
compute bound.

%\mtnote{That sentence is compressed: you want to express at least three
%concepts with it: data- vs compute-intensive workflows; growing role of the
%former on HPC; different requirements of data-intensive workflows on HPC.}. 
As data-intensive workflows and applications are not well supported on HPC, we
motivated the use of the Pilot-Abstraction as an integrating concept between HPC
and data analytics in chapter~\ref{ch:pilot-data-hadoop}. As a result, we 
extended RADICAL-Pilot, an implementation of the Pilot-Abstraction, to support 
Hadoop and Spark on HPC resources ( see \S~\ref{sec:integration_mode}). Through 
our analysis in \S~\ref{ssec:kmeans}, we showed that the RP-YARN reduces the 
time to completion of a data-intensive workflow. Thus, the Pilot-Abstraction 
increases HPC resources utilization in conjunction with emerging data analytics 
frameworks.

%% Chapter 3
Data analytics frameworks offer different abstractions and capabilities and 
selecting a suitable framework is necessary to increase resource utilization 
and reduce the development effort.
%\mtnote{Only resource utilization? Is it more general? Something
%like: "to minimize development effort, adapt the programming model to the
%application requirement and maximize resource utilization"?} 
To this extent in chapter~\ref{ch:task-par}, we investigated the use of three 
task-parallel frameworks---Spark, Dask and RADICAL-Pilot---to implement a range 
of algorithms for MD trajectory analysis. We found that, for embarrassingly 
parallel applications with coarse  grained-tasks, the framework selection does 
not affect the performance. As shown in \S~\ref{sec:psa}, all three frameworks 
showed similar performance and speedups. For fine-grained data-parallel 
applications, a  data-parallel framework is more suitable. \S~\ref{sec:leaflet} 
shows our investigation of a MapReduce style algorithm. Spark and Dask showed 
better performance than RADICAL-Pilot, with Spark offering linear speedups.

In addition to performance gains, developers should take into account usability 
and programmability aspects when selecting a task parallel framework. Spark and 
Dask, as shown in \S~\ref{sec:impl_exp}, required to tune the number of tasks 
to achieve the desired performance (programmability aspect). The programming 
language of the selected framework affects its usability as it may introduce 
overheads due to transferring data between different languages space, e.g., 
between Python and Java for PySpark. Based on our analysis, we provide a 
conceptual framework, in \S~\ref{sec:task_sel_model}, that enables application 
developers to evaluate and select task parallel frameworks with respect to 
application requirements. The conceptual model in conjunction with 
the Pilot-Abstraction provide a methodology to application developers to 
maximize resource utilization while reducing the engineering effort for 
developing and executing data analysis workflows on HPC resources.
%\mtnote{Add references as suggested above but, more
%in general, I would like to see more structure. You have two types of
%contributions: performance (quantitative) and a mix of usability/programmability
%(qualitative). I do not remember well, but reading the chapter those
%contribution were clearly separated. I would make this distinction explicit here
%too and organize the paragraph around it.}

%% Chapter 4
Some data-analysis workflows and applications are both data- and 
compute-intensive and, as such, are not well supported by data-parallel 
frameworks. In chapter~\ref{ch:designs}, we investigated three architecturally 
equivalent designs to execute a data- and compute-intensive workflow. We 
characterized the designs by measuring task execution time 
(\S~\ref{ssec:des1analysis}, \S~\ref{ssec:des2analysis}), resource utilization 
(\S\ref{ssec:exp2}) and workflow time to completion and design overheads 
(\S~\ref{ssec:exp3}). Based on our analysis, we found that late binding data to 
compute nodes, especially in the presence of large amount of data, implies 
copying, replicating or accessing data over network and at runtime. We showed 
that, in contemporary HPC infrastructures, this is too costly both for resource 
utilization and total time to completion. Further, we found that early binding 
of data to compute and equally balancing input across nodes reduced time to 
completion and increased resource utilization. This result provides information 
on the properties a workflow management framework should have to efficiently 
and effectively execute data-driven compute intensive workflows. As a 
consequence, application developers should develop their workflows with a 
framework that allows to early bind and balance data to compute nodes.

%% Chapter 5
Having established our understanding on how to effectively and efficiently
support the execution of data- and compute-intensive workflows on HPC, we 
discussed the design of a campaign manager in chapter~\ref{ch:cmanager}. We 
discuss three actual computational campaigns to elicit the requirements for the 
campaign manager. Based on their requirements, we designed and implemented a 
campaign manager prototype which is domain agnostic and adheres to the building 
blocks approach. We utilized a discrete event simulator (\S~\ref{sec:cm_impl}) 
to simulate the execution of a computational campaign on HPC resources. We 
validated the correctness of design by executing campaigns of different sizes 
on different number of resources (\S~\ref{sec:cm_impl}).

%% Chapter 6
As the campaign manager derives an execution plan for a computational campaign 
and given resources, we investigated three planning algorithms---HEFT, a 
genetic algorithm (GA) and L2FF---to plan the execution of a campaign in 
chapter~\ref{ch:campaigns}. We experimentally compared their plan performance 
in terms of makespan and how sensitive are the plans to resource dynamism and 
workflow runtime uncertainty (\S~\ref{sec:algo_perf_comp}). As computational 
campaigns utilize heterogeneous resources, we showed in 
\S~\ref{ssec:makespan_comp} that HEFT provided at least two times up to an 
order of magnitude better makespan than GA and L2FF. Further, we found that 
plans using HEFT are more sensitive on average than plans using GA and L2FF to 
resource dynamism (\S~\ref{ssec:res_dyn}) and workflow runtime uncertainty 
(\S~\ref{ssec:work_uncert}). Based on our experimental analysis, we discussed 
the algorithmic characteristics that affect the plan performance and 
sensitivity to resource dynamism and workflow runtime estimation uncertainty. 
We conclude by providing a conceptual model where users can select a planning 
algorithm based on their campaign characteristics, the resources they have 
access and their computational objective.

This work will have immediate impact on use cases from earth science domains 
that analyze very high resolution satellite imagery. These sciences want to 
analyze imagery from different calendar years and execute multiple 
workflows for their analysis. In addition, new imagery becomes available in 
a constant low rate stream. Utilizing the work done to develop a campaign 
manager, they will be able to continuously and effectively executing workflows 
that analyze imagery as it becomes available.

\section{Future Work}

The work of this dissertation is invariant of the scientific domain and HPC
resources that a campaign can utilize. As we move forward to support actual
computational campaigns on HPC resources, we see three strands of research and
software development that can be pursued.

First, the campaign manager and specifically its planner, assumes that all
workflows are able to execute on all available resources. In reality,
workflows may have different resource requirements, number and type of
resources, and cannot execute on all resources. We plan to extend the set of
algorithms the planner uses with algorithms that define resource requirements
for workflows and evaluate their makespan performance and sensitivity to
resource dynamism.

Second, all planning algorithms we considered early bind workflows on resources.
Late binding algorithms respond to events during runtime and their execution
plan evolves as the campaign executes. Late binding algorithms can offer the
same performance as early binding algorithms on homogeneous resources and we
believe that they can offer similar performance with HEFT on heterogeneous
resources. As a result, we plan to characterize the performance of a late
binding algorithms and compare it with HEFT's performance.

Finally, the campaign manager prototype does not support the actual execution of
campaigns. As we have an understanding of the requirements to support the
execution of a campaign, we want to investigate what are the requirements to
utilize an actual workflow management framework. RADICAL-EnTK will be our
first choice, as it fits the requirements of the target use cases, and utilizes
a pilot framework as runtime system.
