% !TEX root = main.tex

Computational sciences such as biomolecular
sciences~\cite{cheatham2015impact, dakka2018concurrent}, ecological
sciences~\cite{goncalves2020sealnet, paraskevakos2019workflow} and physics
depending on large scale experiments~\cite{atlas} are benefiting from
executing multiple workflows, with or without dependencies amongst them, to
achieve scientific insight. Users submit multiple workflows for execution and
monitor their execution while managing data transfers and analysis. This way
of execution is called a computational campaign.

Drug discovery campaigns, for example, screen millions of compounds to identify
good drug candidates~\cite{dakka2018concurrent}. Usually, each compound
screening is implemented as a workflow which computes the binding affinity of
the compound. In addition, such applications benefit from utilizing high
performance computing (HPC) resources and consume millions of core-hours of
compute time~\cite{cheatham2015impact} by executing a large number of workflows
over months. Another example of computational campaigns is
quantum chemistry campaigns~\cite{smith2020molssi}. These campaigns usually have
hundreds of workflows, executing for months and on different platforms, such as
workstations, campus and HPC clusters based on workflow
requirements~\cite{smith2020molssi}. Moreover, users prioritize some workflows
over others and remove or add workflows or resources to the campaign during
runtime.

Ecological sciences that utilize very high resolution satellite
imagery~\cite{goncalves2020sealnet}, require the repeated analysis of terabytes
of data from different calendar years to derive time series of ecological
changes. These sciences define workflows with simple or complex
pipelines~\cite{paraskevakos2019workflow} to analyze imagery. These campaigns
analyze imagery from different seasons, calendar years or as new imagery becomes
available by providers. As a result, users frequently define and execute new
workflows to achieve their scientific objective.

High Energy physics experiments, such the Large Hadron Collider ATLAS
experiment~\cite{atlas}, gather and store datasets from detectors or through
simulations in the order of petabytes~\cite{borodin2015big}. These datasets
are then analyzed by executing campaigns with hundreds to thousands workflows
with different timescales and frequencies~\cite{borodin2015big}. For example,
one of the use cases described in Ref.~\cite{borodin2015big} analyzes
thousands of different datasets with workflows running for months every yearly
quarter.

All these use cases define and execute a computational campaign to analyze
data and achieve scientific insight. Although they are different, they share a
number of characteristics and requirements. Some of the most important common
characteristics of these campaigns are:
\begin{inparaenum}[(1)]
    \item they are comprised of O(100) workflows;
    \item workflow requirements are not uniform across a campaign;
    \item use of multiple computational resources;
    \item finite access time to said resources; and
    \item well defined computational objectives.
\end{inparaenum}

Based on the above discussion, it becomes important to efficiently and
effectively support the execution of data-driven computational campaigns as
the first order of execution. This has the potential to increase overall
resource utilization across all resources that are available to a campaign.
Further, increased resource utilization will allow computational scientists to
execute computational campaigns to larger scales, leading to new scientific
discoveries and innovations. As a result, we focus our research on supporting
the efficient execution of data-driven computational campaigns on HPC
resources. It is important to note that our solutions and results are
applicable and extensible to other scientific domains.

\section{Challenges}

The effective and efficient execution of data-driven computational campaigns
poses three main computational challenges:
\begin{inparaenum}[(1)]
    \item support the efficient execution of data-intensive workflows and
    applications;
    \item manage the execution of a computational campaign; and
    \item create an execution plan for executing the campaign.
\end{inparaenum}

Data-intensive workflows and applications are not well supported on HPC
resources, reducing resource utilization and increasing their execution time.
Although, there are several frameworks which provide the necessary abstraction
to execute data analytics at scale, their support and interoperability with HPC
is limited. In addition, there are data analysis workflows which cannot easily
classify as compute- or data-intensive. As these workflows do not always conform
to the HPC or MapReduce style of execution, the capabilities of the workflow
execution engine can affect the efficiency of their execution. In addition, it
becomes important to support the execution of computational campaigns as the
first order of execution \mtnote{not sure what 'first order of execution'
means}.

Supporting the execution of a computational campaign creates a need to design
and offer a campaign manager that supports a diverse set of computational
campaigns. Currently, campaign managers make assumptions about the
resources and the middleware they are utilizing, are monolithic software
systems, and tend to be domain specific. In contrast, a software system
designed and engineered following the building blocks
approach~\cite{turilli2019middleware} is agnostic of the middleware used to
execute the campaign and domain agnostic. This, in turn, will allow the
campaign manager to make no assumptions about the resources on which the
campaign workflows will be mapped and executed.

Finally, the effective and efficient execution of a campaign requires an
execution plan to map workflows on resources. While computational campaigns
can utilize multiple resources concurrently, the effectiveness of a plan is
not guaranteed. Actual HPC resources performance is affected by several
factors and can change significantly during the execution of a campaign. It is
important, then, this execution plan to efficiently map the workflows of the
campaign, so as to increase resource utilization and achieve the objective of
the campaign.

In response to these challenges, we design a campaign manager prototype which
plans and executes computational campaigns on HPC resources. Further, we
provide the necessary extensions and understanding to efficiently and
effectively execute data-driven workflows on HPC resources. Our work is
designed and engineered to be domain and resource agnostic.

\section{Contributions of the Dissertation}

The goal of this research is to advance the state of the art in the execution
of data-intensive scientific computational campaigns. To that extend this
dissertation does the following contributions.

%% First paper contributions:
First, we explore the integration between Hadoop, Spark and HPC resources
utilizing the Pilot-Abstraction~\cite{luckow2012pstar} allowing applications
to manage HPC (e.g., simulations) and data-intensive application stages in a
uniform way. We evaluate the integration by measuring and characterizing the
startup times of the Pilot and tasks. In addition, we measure the performance
of a well-known  algorithm to investigate the runtime trade-offs of a typical
data-intensive application.

%% Second paper contributions:
Second, we investigate three task-parallel frameworks,
Spark~\cite{zaharia2010spark}, Dask~\cite{rocklin2015dask} and
RADICAL-Pilot~\cite{merzky2019using}, and their suitability for implementing
MD trajectory data analysis algorithms. MD trajectories are time series of
atoms/particles positions and velocities, which are analyzed using different
statistical methods to infer certain properties, e.g., the relationship
between distinct trajectories, snapshots of a trajectory etc. As a result,
they can be considered as a representative set of scientific datasets that are
organized as time series. We characterize and explain the behavior of
different molecular dynamics analysis algorithms on these frameworks and we
provide a conceptual basis for comparing the abstractions, capabilities and
performance of these frameworks.

%% Third paper contributions:
Third, we focus on the design of computing frameworks that support the
execution of data-driven compute intensive workflows on HPC resources. We
provide specific design guidelines for supporting data-driven, compute-intense
workflows on high-performance computing resources with a task-based computing
framework. We develop an experiment-based methodology to compare design
performance of alternative designs that does not depend on the considered use
case and computing framework.

Fourth, we design and implement a campaign manager prototype to plan and
execute computational campaigns. We elicit three data analysis use cases to
derive the functional requirements of the campaign manager. We provide a
design and architecture based on the buildings
blocks~\cite{turilli2019middleware} so that the manager is domain and
underlying middleware agnostic.

%%Final Contributions
Lastly, we investigate three planning algorithms,
HEFT~\cite{topcuoglu2002performance}, a genetic
algorithm~\cite{page2005algorithm}, and simple heuristic algorithm, to plan
the execution of a computational campaign on HPC resources. We develop an
experiment-based methodology to compare planning algorithms performance that
does not depend on the considered use case, computing framework and resources
used. In addition, we provide a conceptual framework for selecting planning
algorithms based on the algorithm, campaign and resources characteristics.

\section{Dissertation Organization}
The dissertation is organized into 6 chapters. In
chapter~\ref{ch:pilot-data-hadoop}, we discuss the integration between Hadoop,
Spark and HPC resources by utilizing the Pilot-abstraction.
Chapter~\ref{ch:task-par} discusses the investigation and comparison of three
task-parallel frameworks and their suitability for implementing MD trajectory
analysis algorithms. In Chapter~\ref{ch:designs}, we focus on the design of
computing frameworks that support the execution of workflows on HPC resources
to process large datasets.In Chapter~\ref{ch:cmanager}, we design a campaign
manager (CM) prototype which, given a campaign, an objective, and a set of
resources, can derive an execution plan and simulates the execution of a
campaign. In Chapter~\ref{ch:campaigns}, we investigate algorithms which given
a campaign, an objective and a set of constraints, will derive an execution
plan. The dissertation concludes with Chapter~\ref{ch:conclusions} which
discusses the conclusions of this work and identifies future work and research.

%%%%%%%%%%%%%Old Text%%%%%%%%%%%%%%%%%%%%%%%%%%%%%%%%%%%%%%%%%%%%%

%Compute-based workflows are in the epicenter of scientific computing that utilizes HPC resources.
%These workflows are either defined as a single large application or as an ensemble of compute intensive tasks~\cite{balasubramanian2018harnessing}.
%In both cases, MPI is the most common programming model to implement either the whole application or the tasks of a workflows.
%In addition, HPC resources are being designed to support and optimize the execution of such workflows.

%Data-intensive workflows and applications are associated with a wide variety of characteristics and properties~\cite{fox2014towards,fox2014big}.
%Their complexity and characteristics are distinct from compute intensive applications.
%They often comprise of multiple stages such as, data ingest, pre-processing, feature-extraction and advanced analytics.
%While some of these stages are I/O bound, often with different patterns (random/sequential access), other stages are compute-/memory-bound.
%As a consequence, a diverse set of tools for data processing (e.\,g.\ MapReduce~\cite{dean2004mapreduce}, Spark~\cite{zaharia2010spark}, Dask~\cite{rocklin2015dask}), access to data sources and data formats have emerged and often need to be combined in order to support the end-to-end needs of applications.
%However, this type of tools are not fully supported on HPC resources and it becomes challenging to execute both compute and data intensive applications on the same HPC resources.
%
%There are workflows however which cannot easily classify as compute- or data-intensive.
%In fact, there are classes of scientific applications, such as bio-molecular dynamics~\cite{dror2012biomolecular} and image analysis~\cite{goncalves2020sealnet}, that have strong characteristics of both compute- and data-intensive and do not always conform to the HPC or MapReduce style of execution.
%From a workflow management framework perspective, the design and architectural space is large and the lack of performance analysis make it difficult to select among equivalent implementations.
%

%Missing computational campaign and planning capabilities

%From a design perspective, a promising approach to address these challenges is isolating tasks from execution management.
%Tasks are assumed to be self-contained programs which are executed in the operating system (OS) environment of HPC compute nodes.
%Programs implement the domain-specific functionalities required by use cases while computing frameworks implement resource acquisition, task scheduling, resource binding, and data management.

%Compared to approaches in which tasks are functions or methods, like on Spark~\cite{zaharia2010spark} or Dask~\cite{rocklin2015dask}, a program-based approach offers several benefits as, for example, simplified implementation of execution management, support of general purpose programming models, and separate programming of management and domain-specific functionalities.
%Program-based designs also impose performance limitations, including OS-mediated inter task communication and task spawning overheads, since programs execute as OS processes and do not share a memory space.

%Due to their performance limitations, program-based designs of computing frameworks are best suited to execute compute-intense workflows in which each task requires a certain amount of parallelism and runs from several minutes to hours.
%The emergence of workflows that require heterogeneous, compute-intense tasks to process large amount of data is pushing the boundaries of program-based designs, especially when scale requirements suggest the use of modern HPC infrastructures with large number of CPUs/GPUs and dedicated data facilities.

%Currently, there are software systems ( e.i., campaign managers) that are supporting the execution of computational campaigns~\cite{maeno2008panda,casajus2010dirac,qcfractal,sfiligoi2008glidein,deelman2015pegasus, salim2019balsam}.
%A commonality between current campaign managers is that they are domain specifc and make assumptions about the type of workflows to be executed.
%Another common aspect is that they are build in a monolithic approach, despite their modular design.
%Furthermore, most of the existing campaign managers do not necessarily plan the execution of the campaign, in terms of order and placement of workflows on resources.
%As a result, there is a need to design a campaign manager which, in addition to supporting the above characteristics, will be domain agnostic and provide the planning capabilities.

%Furthermore, the characteristics of the workflows that comprise such applications are different.
%Others gather data either from simulators, such as the ATLAS experiment~\cite{atlas}, or remote sensing, such as satellites~\cite{goncalves2020sealnet}, and require to analyze those data.
%Last but not least are the applications that execute workflows that are simulating systems, analyze produced data and then steer further simulations~\cite{dakka2018concurrent}.

%\subsection*{Paper 1 intro}

%The MapReduce~\cite{mapreduce} abstraction popularized by Apache Hadoop~\cite{hadoop} has been successfully used for many data-intensive applications in different domains~\cite{37684}.
%One important differentiator of Hadoop compared to HPC is the availability of many higher-level abstractions and tools for data storage, transformations and advanced analytics.
%These abstraction typically allow high-level reasoning about data parallelism without the need to manually partition data, manage tasks processing this data and collecting the results, which is required in other environments.
%Within the Hadoop ecosystem, tools like Spark~\cite{Zaharia:2010:SCC:1863103.1863113} have gained popularity by supporting specific data processing and analytics needs and are increasingly used in sciences, e.\,g.\ for DNA sequencing~\cite{Massie:EECS-2013-207}.

%Data-intensive applications are associated with a wide variety of characteristics and properties, as summarized by Fox et\,al.~\cite{bigdata-ogres,bigdata-use-cases-nist}.
%Their complexity and characteristics are fairly distinct from HPC applications.
%For example, they often comprise of multiple stages such as, data ingest, pre-processing, feature-extraction and advanced analytics.
%While some of these stages are I/O bound, often with different patterns (random/sequential access), other stages are compute-/memory-bound.
%Not surprisingly, a diverse set of tools for data processing (e.\,g.\ MapReduce, Spark RDDs), access to data sources (streaming, filesystems) and data formats (scientific data formats (HDF5), columnar formats (ORC and Parquet)) have emerged and they often need to be combined in order to support the end-to-end needs of applications.

%Some applications however, defy easy classification as data-intensive or HPC.
%In fact, there is specific interest in a class of of scientific applications, such as bio-molecular dynamics~\cite{doi:10.1146/annurev-biophys-042910-155245}, that have strong characteristics of both data-intensive and HPC.
%Bio-molecular simulations are now high-performant, reach increasing time scales and problem sizes, and thus generating immense amounts of data.
%The bulk of the data in such simulations is typically trajectory data that is time-ordered set of coordinates and velocity.
%Secondary data includes other physical parameters including different energy components.
%Often times the data generated needs to be analyzed so as to determine the next set of simulation configurations.
%The type of analysis varies from computing the higher order moments, to principal components, to time-dependent variations.

%Bio-molecular simulations are now highly performant, reaching larger problem sizes, and generating immense amounts of data.
%Often, the data generated need to be analyzed and determine the next set of simulation workflows.

%MDAnalysis~\cite{mdanalysis} and CPPTraj~\cite{doi:10.1021/ct400341p} are two tools that evolved to meet the increasing analytics demands of molecular dynamics applications; Ref~\cite{himach} represents an attempt to provide MapReduce based solutions in HPC environments.
%These tools provide powerful domain-specific analytics; a challenge is the need to scale them to high data volumes produced by molecular simulations as well as the coupling between the simulation and analytics parts.
%This points to the need for environments that support scalable data processing while preserving the ability to run simulations at the scale so as to generate the data.
%To the best of our knowledge, there does not exist a solution that provides the integrated capabilities of Hadoop and HPC.
%For example, Cray's analytics platform Urika~\footnote{http://www.cray.com/products/analytics} has Hadoop and Spark running on HPC architecture as opposed to regular clusters, but without the HPC software environment and capabilities.
%However, several applications ranging from bio-molecular simulations to epidemiology models~\cite{network1} require significant simulations interwoven with analysis capabilities such as clustering and graph analytics; in other words some stages (or parts of the same stage) of an application would ideally utilize Hadoop/Spark environments and other stages (or parts thereof) utilize HPC environments.

%MDAnalysis~\cite{michaud2011mdanalysis} and CPPTraj~\cite{roe2013ptraj} are two tools that evolved to meet the increasing analytics demands of molecular dynamics applications;
%Tools like HiMach~\cite{tu2008scalable} represent an attempt to provide MapReduce based solutions in HPC environments.
%These tools support powerful domain-specific data analysis workflows.
%A challenge is the need to scale them to high data volumes as well as couple them with simulation workflows~\cite{balasubramanian2016extasy}.
%To the best of our knowledge, there is no runtime system that provides the integrated capabilities of compute- and data- intensive workflows and applications.

%Frameworks for parallel data analysis have been created by the High Performance Computing (HPC) and Big Data communities~\cite{kamburugamuve2017anatomy}.
%MPI is the most used programming model for HPC resources.
%It assumes a SPMD execution model where each process executes the same program.
%It is highly optimized for high-performance computing and communication, which along with synchronization need explicit implementation.
%Big Data frameworks utilize higher-level MapReduce~\cite{dean2004mapreduce} programming models avoiding explicit implementations of communication.
%In addition, the MapReduce~\cite{dean2004mapreduce} abstraction makes it easy to exploit data-parallelism as required by many analysis workflows.

%Task-parallel applications involve partitioning a workload into a set of self-contained units of work.
%Based on the application, these tasks can be independent, have no inter-task communication, or coupled with varying degrees of data dependencies.
%Big Data applications exploit task parallelism for data-parallel parts (e.\,g., \texttt{map} operations), but also require coupling, for computing aggregates (the \texttt{reduce} operation).
%The MapReduce~\cite{dean2004mapreduce} abstraction popularized this execution pattern.
%Typically, a reduce operation includes shuffling intermediate data from a set of nodes to node(s) where the reduce executes.

%Spark~\cite{zaharia2010spark} and Dask~\cite{rocklin2015dask} are two Big Data frameworks.
%Both provide MapReduce abstractions and are optimized for parallel processing of large data volumes, interactive analytics and machine learning.
%Their runtime engines can automatically partition data, generate parallel tasks, and execute them on a cluster.
%In addition, Spark offers in-memory capabilities allowing caching data that are read multiple times, making it suited for interactive analytics and iterative machine learning algorithms.
%Dask also provides a MapReduce API (Dask Bags).
%Furthermore, Dask's API is more versatile, allowing custom workflows and parallel vector/matrix computations.
%Over the past decades, the High Performance Distributed Computing (HPDC) community has made significant advances in addressing resource and workload management on heterogeneous resources.
%For example, the concept of multi-level scheduling~\cite{1392910} as manifested in the decoupling of workload assignment from resource management using the concept of intermediate container jobs (also referred to as Pilot-Jobs~\cite{pstar12}) has been adopted for both HPC and Hadoop.
%Multi-level scheduling is a critical capability for data-intensive applications as often only application-level schedulers can be aware of the localities of the data sources used by a specific application.
%This motivated the extension of the Pilot-Abstraction to Pilot-Data~\cite{pilot-data-jpdc-2014} to form the central component of a resource management middleware.

%In this paper, we explore the integration between Hadoop and HPC resources utilizing the Pilot-Abstraction allowing application to manage HPC (e.\,g.\ simulations) and data-intensive application stages in a uniform way.
%We propose two extensions to RADICAL-Pilot: the ability to spawn and manage Hadoop/Spark clusters on HPC infrastructures on demand (Mode I), and to connect and utilize Hadoop and Spark clusters for HPC applications (Mode II).
%Both extensions facilitate the complex application and resource management requirements of data-intensive applications that are best met by a best-of-bread mix of Hadoop and HPC.
%By supporting these two usage modes, RADICAL-Pilot dramatically simplifies the barrier of deploying and executing HPC and Hadoop/Spark side-by-side.

%We utilize the Pilot-Abstraction~\cite{luckow2012pstar} allowing the application to manage HPC and data-intensive application stages in a uniform way.
%We propose two extensions to RADICAL-Pilot~\cite{merzky2018design}, a Pilot-Job~\cite{luckow2012pstar} runtime system designed for implementing task-parallel applications on HPC:
%\begin{inparaenum}[(i)]
%    \item the ability to spawn and manage Hadoop/Spark clusters on HPC infrastructures on demand (Mode I),
%    \item and to connect and utilize Hadoop and Spark clusters for HPC applications (Mode II)
%\end{inparaenum}.
%Both extensions facilitate the complex application and resource management requirements of data-intensive applications.
%By supporting these two usage modes, RADICAL-Pilot dramatically simplifies the barrier of deploying and executing HPC and Hadoop/Spark side-by-side.

%In addition to Spark and Dask, we investigate RADICAL-Pilot~\cite{merzky2018design}.

%MD trajectories are time series of atoms/particles positions and velocities, which are analyzed using different statistical methods to infer certain properties, e.\,g. the relationship between distinct trajectories, snapshots of a trajectory etc.
%As a result, they can be considered as a representative set of scientific datasets that are organized as time series and their analysis algorithms.

%\subsection*{Paper 2 intro}
%Frameworks for parallel data analysis have been created by the High Performance Computing (HPC) and Big Data communities~\cite{fox-2017}.
%MPI is the most used programming model for HPC resources.
%It assumes a SPMD execution model where each process executes the same program.
%It is highly optimized for high-performance computing and communication, which along with synchronization need explicit implementation.
%Big Data frameworks utilize higher-level MapReduce~\cite{mapreduce} programming models avoiding explicit implementations of communication.
%In addition, the MapReduce~\cite{mapreduce} abstraction makes it easy to exploit data-parallelism as required by many analysis applications.
%Several recent publications applied HPC techniques to advance traditional Big Data applications and Big Data frameworks~\cite{fox-2017}.

%Task-parallel applications involve partitioning a workload into a set of self-contained units of work.
%Based on the application, these tasks can be independent, have no inter-task communication, or coupled with varying degrees of data dependencies.
%Big Data applications exploit task parallelism for data-parallel parts (e.\,g., \texttt{map} operations), but also require coupling, for computing aggregates (the \texttt{reduce} operation).
%The MapReduce~\cite{mapreduce} abstraction popularized this execution pattern.
%Typically, a reduce operation includes shuffling intermediate data from a set of nodes to node(s) where the reduce executes.
%There is a recognized need to optimize communication intensive parts of Big Data frameworks using established HPC techniques for interprocess, e.\,g. shuffle operations~\cite{rdma-spark} and other forms of communication~\cite{hpc-abds,twisterNet}.

%Spark~\cite{zaharia2010spark} and Dask~\cite{rocklin2015dask} are two Big Data frameworks.
%Both provide MapReduce abstractions and are optimized for parallel processing of large data volumes, interactive analytics and machine learning.
%Their runtime engines can automatically partition data, generate parallel tasks, and execute them on a cluster.
%In addition, Spark offers in-memory capabilities allowing caching data that are read multiple times, making it suited for interactive analytics and iterative machine learning algorithms.
%Dask also provides a MapReduce API (Dask Bags).
%Furthermore, Dask's API is more versatile, allowing custom workflows and parallel vector/matrix computations.

%In this paper, we investigate the data analysis requirements of Molecular Dynamics (MD) applications.
%MD simulations are significant consumers of computing cycles, producing immense amounts of data.
%A typical $\mu sec$ MD simulation of physical system of $O(100k)$ atoms can produce from $O(10)$ to $O(1000)$ GBs of data~\cite{cheatham2015impact}.
%In addition to being the prototypical HPC application, there is increasingly a need for the analysis to be integrated with simulations and drive the next stages of execution~\cite{balasubramanian2016extasy}.
%The analysis phase must be performed quickly and efficiently in order to steer the simulations.

%We investigate three task-parallel frameworks and their suitability for implementing MD trajectory analysis algorithms.
%In addition to Spark and Dask, we investigate RADICAL-Pilot~\cite{rp-jsspp18}, a Pilot-Job~\cite{pstar12} framework designed for implementing task-parallel applications on HPC.
%We utilize MPI4py~\cite{mpi4py_paper} to provide MPI equivalent implementations of the algorithms.
%The task-parallel implementations performance and scalability compared to MPI is the basis of our analysis.
%MD trajectories are time series of atoms/particles positions and velocities, which are analyzed using different statistical methods to infer certain properties, e.\,g. the relationship between distinct trajectories, snapshots of a trajectory etc.
%As a result, they can be considered as a representative set of scientific datasets that are organized as time series and their analysis algorithms.

%\subsection*{Paper 3 intro}
%A growing number of scientific domains are adopting workflows that use multiple analysis algorithms to process a large number of images.
%The volume and scale of data processing justifies the use of parallelism, tailored programming models and high performance computing (HPC) resources.
%While these features create a large design space, the lack of architectural and performance analyses makes it difficult to chose among functionally equivalent implementations.

%In this paper we focus on the design of computing frameworks that support the execution of heterogeneous tasks on HPC resources to process large imagery datasets.
%These tasks may require one or more CPUs and GPUs, implement diverse functionalities and execute for different amounts of time.
%Typically, tasks have data dependencies and are therefore organized into workflows.
%Due to task heterogeneity, executing workflows poses the challenges of effective scheduling, correct resource binding and efficient data management.
%HPC infrastructures exacerbate these challenges by privileging the execution of single, long-running jobs.

%Queuing networks will be of limited use because they require from the user to provide a queuing network equivalent to the campaign.
%In the case the campaign contains only independent workflows, a single queuing system with multiple servers would be sufficient, but a campaign with complex dependencies between workflows may require expertise outside the user's domain to define the equivalent queuing network.
%Domain specific languages approaches either require description of the resource usage of workflows~\cite{carothers2017durango}, or execute part of the campaign to obtain an execution ``skeleton'' of the campaign~\cite{maheshwari2016workflow}.
%When executing a campaign, workflows may require days to execute to obtain execution time information, and users rarely know the resource usage of their workflows to provide accurate enough information.
%In addition, the workflows of a campaign may be different and executing some of them may not provide any information about the execution of others.

%Currently, campaign managers are making assumptions about the resources and the middleware they are utilizing, are monolithic software systems, and tend to be domain specific.
%The CM we propose to prototype will avoid these three limitations.
%Our CM will support multiple use case from different domains, such as molecular dynamics and earth sciences.
%As a result, it will be domain agnostic.
%In addition, our CM will also be designed by following the building blocks approach~\cite{turilli2019middleware}.
%In this way, it will be agnostic of the system used to manage the execution of the campaign workflows.
%This, in turn, will allow our CM to make no assumptions about the resources on which the campaign workflows will be mapped and executed.

%Execution planning for workflows are provided by several workflow execution systems, such as Pegasus~\cite{deelman2015pegasus}, and ASKALON~\cite{fahringer2005askalon}.
%Campaign management systems, such as PanDA~\cite{maeno2008panda}, do not provide a campaign planning feature.
%QCFractal~\cite{qcfractal} offers some form of planning by allowing user to specify the priority of a workflow in a campaign, but this planning does not take into account the makespan of the campaign.

%Over the past decades, the HPC community has made significant advances in addressing resource and workload management on heterogeneous resources.
%For example, the concept of multi-level scheduling~\cite{berman1996application} as manifested in decoupling workload placement from resource management using intermediate container jobs (also known as Pilot-Jobs~\cite{luckow2012pstar}) has been adopted for both HPC and Hadoop.
%Multi-level scheduling is a critical capability for data-intensive applications, as often only application-level schedulers are aware of data locality of the data used by an application.

%A growing number of scientific domains are adopting workflows that use multiple analysis algorithms to process large datasets.
%The volume and scale of data processing justifies the use of parallelism, tailored programming models and high performance computing (HPC) resources.
%While these features create a large design space, the lack of architectural and performance analyses makes it difficult to chose among functionally equivalent implementations.

%Computational campaigns enact an execution plan to allow users to achieve a computational objective under given requirements and constraints.
%A computational objective is a set of values selected by the user for a set of metrics, e.g., time to completion and throughput, which can be represented as an objective function.
%Requirements describe the minimum amount and type of resources needed to execute each workflow of the campaign, while constraints are the conditions that bound the execution, including but not limited to, resource availability, capacity or costs.
%A plan describes a sequence of actions that solves the objective function as, for example, selecting, acquiring and configuring resources, and establishing the execution order of workflows on resources.

%Planning and enacting the execution of a campaign poses four main challenges:
%\begin{inparaenum}[(i)]
%    \item evaluating the makespan of a campaign on heterogeneous and dynamic resources;
%    \item the conditions under which an execution plan performs better compared to a random resource selection;
%    \item determining a campaign execution plan on available resources that satisfies the given objective function, requirements and constraints of a campaign, while accounting for resource dynamism; and
%    \item adapting the plan in case of deviation from the objective achievement.
%\end{inparaenum}

%There are several methods and algorithms to calculate and optimize the makespan of a workflow~\cite{lu2019review}, including list-heuristics~\cite{dong2006scheduling,list_sched_wiki}, queuing networks~\cite{yao2019throughput,bao2019performance}, domain specific languages~\cite{carothers2017durango,maheshwari2016workflow}, genetic algorithms~\cite{karla2015review} and machine learning~\cite{witt2019predictive,pumma2017runtime}.