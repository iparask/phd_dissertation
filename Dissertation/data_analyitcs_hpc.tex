% !TEX root = main.tex
\label{ch:data_hpc}
%Frameworks for parallel data analysis have been created by the High Performance Computing (HPC) and Big Data communities~\cite{kamburugamuve2017anatomy}.
%MPI is the most used programming model for HPC resources.
%It assumes a SPMD execution model where each process executes the same program.
%It is highly optimized for high-performance computing and communication, which along with synchronization need explicit implementation.
%Big Data frameworks utilize higher-level MapReduce~\cite{dean2004mapreduce} programming models avoiding explicit implementations of communication.
%In addition, the MapReduce~\cite{dean2004mapreduce} abstraction makes it easy to exploit data-parallelism as required by many analysis workflows.

Frameworks for parallel data analysis have been created by the High Performance Computing (HPC) and Big Data communities~\cite{kamburugamuve2017anatomy}.
Until now MPI is the most used programming model.
It assumes a SPMD execution model where each process executes the same program.
It is highly optimized for high-performance computing and communication, which along with synchronization need explicit implementation.
Big Data frameworks utilize higher-level MapReduce~\cite{dean2004mapreduce} programming models avoiding explicit implementations of communication.
In addition, the MapReduce~\cite{dean2004mapreduce} abstraction makes it easy to exploit data-parallelism as required by many analysis workflows.

Task-parallel applications involve partitioning a workload into a set of self-contained units of work.
Based on the application, these tasks can be independent, have no inter-task communication, or coupled with varying degrees of data dependencies.
Big Data applications exploit task parallelism for data-parallel operations (e.\,g., \texttt{map}), but also require coupling, for computing aggregates (\texttt{reduce}).
The MapReduce~\cite{dean2004mapreduce} abstraction popularized this execution pattern.
Typically, a reduce operation includes shuffling intermediate data from a set of nodes to node(s) where the reduce executes.

Spark~\cite{zaharia2010spark} and Dask~\cite{rocklin2015dask} are two Big Data frameworks.
Both provide MapReduce abstractions and are optimized for parallel processing of large data volumes, interactive analytics and machine learning.
Their runtime engines can automatically partition data, generate parallel tasks, and execute them on a cluster.
In addition, Spark offers in-memory capabilities allowing caching data that are read multiple times, making it suited for interactive analytics and iterative machine learning algorithms.
Dask also provides a MapReduce API (Dask Bags).
Furthermore, Dask's API is more versatile, allowing custom workflows and parallel vector/matrix computations.

In \S~\ref{sec:pilot-data-hadoop}, we explore the integration between Hadoop~\cite{hadoop}, and Spark~\cite{zaharia2010spark} and HPC resources.
We utilize the Pilot-Abstraction~\cite{luckow2012pstar} allowing the application to manage HPC and data-intensive application stages in a uniform way.
We explore two extensions to RADICAL-Pilot~\cite{merzky2018design}, a Pilot-Job~\cite{luckow2012pstar} runtime system designed for implementing task-parallel applications on HPC: 
\begin{inparaenum}[(i)]
    \item the ability to spawn and manage Hadoop/Spark clusters on HPC infrastructures on demand (Mode I),
    \item and to connect and utilize Hadoop and Spark clusters for HPC applications (Mode II)
\end{inparaenum}.
Both extensions facilitate the complex application and resource management requirements of data-intensive applications.
By supporting these two usage modes, RADICAL-Pilot dramatically simplifies the barrier of deploying and executing HPC and Hadoop/Spark side-by-side.

In \S~\ref{sec:task-par}, we investigate three task-parallel frameworks, Spark, Dask and RADICAL-Pilot, and their suitability for implementing MD trajectory analysis algorithms.
We also utilize MPI4py~\cite{dalcin2005mpi} to provide MPI equivalent implementations of the algorithms.
The task-parallel implementations performance and scalability compared to MPI is the basis of our analysis.
MD trajectories are time series of atoms/particles positions and velocities, which are analyzed using different statistical methods to infer certain properties, e.\,g. the relationship between distinct trajectories, snapshots of a trajectory etc.
As a result, they can be considered as a representative set of scientific datasets that are organized as time series and their analysis algorithms. 
The section makes the following contributions: 
\begin{inparaenum}[i)]
    \item it characterizes and explains the behavior of different MDAnalysis algorithms on these frameworks, and
    \item provides a conceptual basis for comparing the abstraction, capabilities and performance of these frameworks.
\end{inparaenum}


\section{Integrating Hadoop and Spark with HPC workload management system}
\label{sec:pilot-data-hadoop}
Our approach is to design a common software environment while attempting to be agnostic of specific hardware infrastructures, technologies and trends.
%In this section, we provide background information and comparative information on system abstractions, resource management and interoperability in HPC and Hadoop.
Hadoop~\cite{hadoop} has evolved to become the standard implementation of the MapReduce abstraction on top of the Hadoop filesystem and the YARN resource management.
In fact, over the past years, Hadoop evolved to a general purpose cluster computing framework suited for data-intensive applications in industry~\cite{luckow2015automotive} and sciences~\cite{jha2014tale}.

HPC and Hadoop originated from the need to support different kinds of applications: compute-intensive applications in the case of HPC, and data-intensive in the case of Hadoop.
Not surprisingly, they follow different design paradigms: In HPC environments, storage and compute are connected by a high-end network (e.\,g.\ Infiniband) with capabilities such as RDMA; Hadoop co-locates both.
HPC infrastructures introduced parallel filesystems, such as Lustre, PVFS or GPFS, to meet the increased I/O demands of data-intensive applications and archival storage and to address the need for retaining large volumes of primary simulation output data.
The parallel filesystem model of using large, optimized storage clusters exposing a POSIX compliant rich interface and connecting it to compute nodes via fast interconnects works well for compute-bound tasks.
It has however, some limitations for data-intensive, I/O-bound workloads that require a high sequential read/write performance.
Various approaches for integrating parallel filesystems, such as Lustre and PVFS, with Hadoop emerged~\cite{kulkarni2013hadoop,tantisiriroj2011duality}, which yielded good results in particular for medium-sized workloads.

While Hadoop simplified the processing of vast volumes of data, it has limitations in its expressiveness as pointed out by various authors~\cite{yelick2011magellan,isard2007dryad}.
The complexity of creating sophisticated applications such as iterative machine learning algorithms required multiple MapReduce jobs and persistence to HDFS after each iteration.
This is lead to several higher-level abstractions for implementing sophisticated data pipelines.
Examples of such higher-level execution management frameworks for Hadoop are: Spark~\cite{zaharia2010spark}, Apache Flink~\cite{flink}, Apache Crunch~\cite{crunch} and Cascading~\cite{cascading}.

The most well-known emerging processing framework in the Hadoop ecosystem is Spark~\cite{zaharia2010spark}.
In contrast to MapReduce, it provides a richer API, more language bindings and a novel memory-centric processing engines, that can utilize distributed memory and can retain resources across multiple task generation.
Spark's \emph{Reliable Distributed Dataset (RDD)} abstraction provides a powerful way to manipulate distributed collections stored in cluster nodes' memory.
Spark is increasingly used for building complex data workflows and advanced analytic tools, such as MLLib~\cite{mllib} and SparkR.
Although the addition/development of new and higher-level execution frameworks addressed some of the problems of data processing, it introduced the problem of heterogeneity of access and resource management.

Hadoop originally provided a rudimentary resource management system.
The YARN scheduler~\cite{vavilapalli2013apache} provides a robust application-level scheduler framework addressing the increased requirements with respect to applications and infrastructure: 
%more complex data localities (memory, SSDs, disk, rack, datacenter), long-lived services, periodic jobs, interactive and batch jobs need to be supported on the same environment.
In contrast, to traditional batch schedulers, YARN is optimized for data-intensive environments supporting data-locality and the management of a large number of fine-granular tasks.
%(found in data-parallel applications).

While YARN manages system-level resources, applications and runtimes have to implement an application-level scheduler that optimizes their specific resource requirements, e.\,g.\ with respect to data locality. This application-level scheduler is referred to as {\it Application Master} and is responsible for allocating resources -- the so called containers -- for the applications and to execute tasks in these containers.
Data locality, e.\,g.\ between HDFS blocks and container locations, need to managed by the Application Master by requesting containers on specific nodes/racks.

Managing resources on top of YARN is associated with several challenge: while fine-grained, short-running tasks as found in data-parallel MapReduce applications are well supported, other workload characteristics are less well supported,such as parallel MPI applications and long-running applications.
%To achieve interoperability and integration between Hadoop and HPC, it is essential to consider a more diverse set of workloads on top of  YARN.
\section{State of the art}
To achieve interoperability between Hadoop and HPC, several frameworks explore the usage of Hadoop on HPC resources, e.\,g., Hadoop on Demand~\cite{hod}, JUMMP~\cite{moody2013jummp}, MagPie~\cite{chu2015magpie}, MyHadoop~\cite{krishnan2011myhadoop}, MyCray~\cite{mycray}.
While these frameworks can spawn and manage Hadoop clusters many challenges with respect to optimizing configurations and resource usage including the use of available SSDs for the shuffle phase, of parallel filesystems and of high-end network features, e.\,g.\ RDMA~\cite{rahman2014homr} remain.
Further, these approaches do not address the need for interoperability between HPC and Hadoop application stages.

A particular challenge for Hadoop on HPC deployment is the choice of storage and filesystem backend.
Typically, for Hadoop local storage is preferred; nevertheless, in some cases, e.\,g.\ if many small files need to processed, random data access is required or the number of parallel tasks is low to medium, the usage of Lustre or another parallel filesystem can yield in a better performance.
For this purpose, many parallel filesystems provide a special client library, which improves the interoperability with Hadoop; it limits however data locality and the ability for the application to optimize for data placements.
% since applications are commonly not aware of the complex storage hierarchy.
%Another interesting usage mode is the use of Hadoop as active archival storage -- in particular, the newly added HDFS heterogeneous storage support is suitable for supporting this use case.

Another challenge is the integration between both HPC and Hadoop environments.
Rather than preserving HPC and Hadoop ``environments'' as software silos, there is a need for an approach that integrates them. 
We utilize the Pilot-Abstraction as an unifying concept to efficiently support the integration, and not just the interoperability between HPC and Hadoop.
By utilizing the multi-level scheduling capabilities of YARN, the Pilot-Abstraction can efficiently manage Hadoop cluster resources providing the application with the necessary means to reason about data and compute resources and allocation.
On the other side, we show, how the Pilot-Abstraction can be used to manage Hadoop applications on HPC environments.

The Pilot-Abstraction~\cite{luckow2012pstar} has been successfully used in HPC environments for supporting a diverse set of task-based workloads on distributed resources.
A Pilot-job is a placeholder job that is submitted to the resource management system representing a container for a dynamically determined set of compute tasks.
Pilot-jobs are a well-known example of multi-level scheduling, which is often used to separate system-level resource and user-level workload management.
The Pilot-Abstraction defines the following entities:
\begin{inparaenum} [(i)]
    \item a Pilot-Compute allocates a set of computational resources (e.\,g.\,cores);
    \item  a Compute-Unit (CU) as a self-contained piece of work represented as executable that is submitted to the Pilot-job.
\end{inparaenum}
%A CU can have data dependencies, i.\,e.\ a set of files that need to be available when executing the CU.
%A workflow typically consists of a set of dependents CUs.
The Pilot-Abstraction has been implemented within RADICAL-Pilot~\cite{merzky2018design}.
%The interoperability layer is SAGA~\cite{merzky2015saga}, which is used for accessing the resource management system (e.\,g.\ SLURM, Torque and SGE) and for file transfers.
%SAGA is a lightweight interface that provides standards-based interoperable capabilities to the most commonly used functionalities required to develop distributed applications, tools and services.


\section{Data analytics and HPC modes of integration}
\label{ssec:integration_mode}
%An important motivation of our work is to provide advanced and scalable data analysis capabilities for existing high-performance applications (e.g., large-scale molecular dynamics simulations).
%This requires adding data-intensive analysis while preserving high-performance computing capabilities.
Having established the potential of the Pilot~-Abstraction for a range of high-performance applications~\cite{treikalis2016repex,ragothaman2014developing,ko2014numerical}, we use it as the starting point for integrated high-performance compute and data-intensive analysis.
As depicted in Figure~\ref{fig:figures_hadoop-on-hpc-viceverse}, there are at least two different usage modes to consider:
\begin{compactenum}[(i)]
    \item Mode I: Running Hadoop/Spark applications on HPC environments (Hadoop on HPC),
    \item Mode II: Running HPC applications on YARN clusters (HPC on Hadoop).
\end{compactenum}

\begin{figure}[t]
    \centering
    \includegraphics[width=.95\textwidth]{figures/data_analytics_hpc/hpc_hadoop/hadoop-on-hpc-viceverse.pdf}
    \caption{Hadoop and HPC Interoperability modes\label{fig:figures_hadoop-on-hpc-viceverse}}
\end{figure}

Mode I is critical to support traditional HPC environments (e.\,g., the majority of XSEDE~\cite{xsede} resources) so as to support applications with both compute and data requirements.
Mode II is important for cloud environments (e.\,g.\ Amazon's Elastic MapReduce, Microsoft's HDInsight) and an emerging class of HPC machines with new architectures and usage modes, such as Wrangler~\cite{wrangler} that support Hadoop natively.
For example, Wrangler supported dedicated Hadoop environments (based on Cloudera Hadoop 5.3) via a reservation mechanism.

In the following we discuss a set of tools for supporting both of these usage modes:
In section~\ref{sssec:saga_hadoop} we present SAGA-Hadoop, a light-weight, easy-to-use tool for running Hadoop on HPC (Mode I).
We then discuss, the integration of Hadoop and Spark runtimes into RADICAL~-Pilot, which enables both the interoperable use of HPC and Hadoop, as well as the integration of HPC and Hadoop applications (Mode I and II) (Section~\ref{sssec:rp-impl} to~\ref{sssec:rp_spark}).
Using these new capabilities, applications can seamlessly connect HPC stages (e.\,g.\ simulation stages) with analysis staging using the Pilot-Abstraction to provide unified resource management.

\subsection{SAGA-Hadoop: Supporting Hadoop/Spark on HPC}
\label{sssec:saga_hadoop}

SAGA-Hadoop~\cite{saga-hadoop} is a tool for supporting the deployment of Hadoop and Spark on HPC resources (Mode I in Figure~\ref{fig:figures_hadoop-on-hpc-viceverse}).
Using SAGA-Hadoop an applications written for YARN (e.\,g.\ MapReduce) or Spark (e.\,g. PySpark, DataFrame and MLlib applications) can be executed on HPC resources.

Figure~\ref{fig:saga-hadoop} illustrates the architecture of SAGA-Hadoop.
SAGA-Hadoop uses SAGA~\cite{merzky2015saga} to spawn and control Hadoop clusters inside an environment managed by an HPC scheduler.
%, such as PBS, SLURM or SGE, or clouds.
SAGA is used for dispatching a bootstrap process that generates the necessary configuration files and starting Hadoop.
The specifics of the Hadoop framework (i.\,e.\ YARN and Spark) are encapsulated in a Hadoop framework plugin (also referred to as adaptors).
SAGA-Hadoop delegates tasks, such as the download, configuration and start of a framework to this plugin.
In the case of YARN, the plugin is then responsible for launching YARN's Resource and Node Manager processes; in the case of Spark, the Master and Worker processes.
This architecture is extensible as new frameworks, e.\,g.\ Flink, can easily be added.

\begin{figure}[t]
    \centering
    \includegraphics[width=.95\textwidth]{figures/data_analytics_hpc/hpc_hadoop/pilot-abds.pdf}
    \caption{SAGA-Hadoop for HPC and Cloud Infrastructures.}
    \label{fig:saga-hadoop}
\end{figure}

While nearly all Hadoop frameworks (e.\,g.\ MapReduce and Spark) support YARN for resource management, Spark provides a standalone cluster mode, which is more efficient in particular on dedicated resources.
Thus, a special adaptor for Spark is provided.
Once the cluster is setup, users can submit applications using SAGA's API that allows them to start and manage YARN or Spark application processes.

While SAGA-Hadoop provides the interoperability between YARN and HPC resources by treating YARN as a substitute for common HPC schedulers, the integration of YARN and HPC applications or application stages remains challenging.
As a consequence, we explored the usage of the Pilot~-Abstraction~\cite{luckow2012pstar}, through RADICAL~-Pilot~\cite{merzky2019using}, to enable the integration between these different application types.

\subsection{RADICAL~-Pilot and YARN Overview}
\label{sssec:rp-impl}
With the introduction of YARN, a broader set of applications can be executed within Hadoop clusters than earlier.
However, developing and deploying YARN applications potentially side-by-side with HPC applications remains a difficult task.
Established abstractions that are easy-to-use while enabling the user to reason about compute and data resources across infrastructure types (i.\,e.\ Hadoop, HPC and clouds) are missing. 

Schedulers such as YARN effectively facilitate application-level scheduling, the development efforts for YARN applications are high. YARN provides a low-level abstraction for resource management, e.g., a Java API and protocol buffer specification.
Typically interactions between YARN and the applications are much more complex than the interactions between an application and a HPC scheduler.
Further, applications must be able to run on a dynamic set of resources; YARN e.\,g.\ can preempt containers in high-load situations.
Data/compute locality need to be manually managed by the application scheduler by requesting resources at the location of an file chunk.
Also, allocated resources (the so called YARN containers) can be preempted by the scheduler.

To address these shortcomings, various frameworks that aid the development of YARN applications have been proposed: Llama~\cite{llama} offers a long-running application master for YARN designed for the Impala SQL engine.
Apache Slider~\cite{apache-slider} supports long-running distributed application on YARN with dynamic resource needs allowing applications to scale to additional containers on demand.
While these frameworks simplify development, they do not address concerns such as interoperability and integration of HPC/Hadoop.
Integrating YARN and RADICAL~-Pilot (RP) allows applications to run HPC and YARN applications on HPC resources concurrently.
%In the following, we explore the integration of YARN into the RADICAL~-Pilot (RP) framework.
%This approach allows applications to run HPC and YARN application parts side-by-side.

% RADICAL-Pilot Overview
Figure~\ref{fig:comp_rp_arch} illustrates the architecture of RADICAL~-Pilot and the components that were extended for YARN.
The figure on the left shows the macro architecture of RADICAL~-Pilot; the figure on the right shows the architecture of the Pilot-Agent which is a critical functional component.
RADICAL~-Pilot consists of a client module with the Pilot-Manager and Unit-Manager and a set of RADICAL~-Pilot-Agents running on resources.
The Pilot-Manager is the central entity responsible for managing the lifecycle of a set of Pilots: Pilots are described using a Pilot description, which contains the resource requirements of the Pilot and is submitted to the Pilot-Manager.
The Pilot-Manager submits the placeholder job that will run the RADICAL~-Pilot-Agent via the Resource Management System using the SAGA-API (steps P.1-P.2).
Subsequently, the application workload (the Compute-Units) is managed by the Unit-Manager and the RADICAL~-Pilot-Agent (steps U.1-U.7)~\cite{merzky2019using}.

\begin{figure}
    \centering
    \includegraphics[width=0.95\textwidth]{figures/data_analytics_hpc/hpc_hadoop/rp-architecture-yarn.pdf}
    \caption{RADICAL-Pilot and YARN Integration.\label{fig:comp_rp_arch}}
%\textbf{RADICAL-Pilot and YARN Integration:} There are two main interactions between the application and RADICAL~-Pilot -- the management of Pilots (P.1-P.2) and the management of Compute Units (U.1-U.7).
%        All YARN specifics are encapsulated in the RADICAL-Pilot-Agent.\label{fig:comp_rp_arch}
\end{figure}

The RADICAL~-Pilot-Agent has a modular and extensible architecture and consists of the following components: the Agent Execution Component, the Heartbeat Monitor, Agent Update Monitor, Stage In and Stage Out Workers.
The main integration of YARN and RP is done in the Agent Execution Component.
This component consist of four sub-components:
\begin{inparaenum}[a)]
    \item the \textit{scheduler} is responsible for monitoring the resource usage and assigning CPUs/GPUs to Compute Units;
    \item the \textit{Local Resource Manager} (LRM) interacts with the batch system and communicates to the Pilot and Unit Managers the available number of computing reosurces and how they are distributed;
    \item the \textit{Task Spawner/ Executor} configures the execution environment, executes and monitors each unit; 
    \item and the \textit{Launch Method} encapsulates the environment specifics for executing an application, e.\,g.\ the usage of \texttt{mpiexec} for MPI applications, machine-specific launch methods (e.g. \texttt{aprun} on Cray machines) or the usage of YARN.
\end{inparaenum}
After the Task Spawner completes the execution of a unit, it collects the exit code, standard input and output, and instructs the scheduler about the freed cores.

\subsection{Integration of RADICAL~-Pilot and YARN}
\label{sssec:rp-yarn}

There are two integration options for RADICAL~-Pilot and YARN: (i) Integration on Pilot-Manager level, via a SAGA adaptor, and (ii) integration on the RADICAL~-Pilot-Agent level.
The first approach is associated with several challenges: firewalls typically prevent the communication between external machines and a YARN clusters.
A YARN application is not only required to communicate with the resource manager, but also with the node managers and containers; further, this approach would require significant extension to the Pilot-Manager, which currently relies on the SAGA Job API for launching and managing Pilots.
Capabilities like the on-demand provisioning of a YARN cluster and the complex application-resource management protocol required by YARN are difficult to abstract behind the SAGA API.

The second approach encapsulated YARN specifics on resource-level.
If required, a YARN cluster is decentrally provisioned.
Units are scheduled and submitted to the YARN cluster via the Unit-Manager, the MongoDB-based communication protocol and the RADICAL~-Pilot-Agent scheduler.
By integrating at the RADICAL~-Pilot-Agent level, RADICAL~-Pilot supports both Mode I and II as outlined in Figure~\ref{fig:figures_hadoop-on-hpc-viceverse}.

As illustrated in Figure~\ref{fig:comp_rp_arch}, in the first phase (step P.1 and P.2) the RADICAL~-Pilot-Agent is started on the remote resource using SAGA.
In Mode I, during the launch of the RADICAL~-Pilot-Agent the YARN cluster is spawned on the allocated resources (Hadoop on HPC); in Mode II the RADICAL~-Pilot-Agent will just connect to a YARN cluster running on the machine of the RADICAL~-Pilot-Agent.
Once the RADICAL~-Pilot-Agent has been started, it is ready to accept Compute Units submitted via the Unit-Manager (step U.1).
The Unit-Manager queues new Compute Units using a shared MongoDB instance (step U.2).
The RADICAL~-Pilot-Agent periodically checks for new Compute Units (U.3) and queues them inside the scheduler (U.4).
The execution of the Compute Unit is managed by the Executor (step U.6 and U.7).
%In the following, we describe how these components have been extended to support YARN.

The \emph{Local Resource Manager (LRM)} provides an abstraction to local resource details for other components of the RADICAL~-Pilot-Agent.
The LRM evaluates the environment variables provided by the resource management systems to obtain information, such as the number of cores per node, memory and the assigned nodes.
This information can be accessed through the Resource Manager's REST API.
As described, there are two deployment modes.
In Mode I (Hadoop on HPC), during the initialization of the RADICAL~-Pilot-Agent, the LRM setups the HDFS and YARN daemons.
First, the LRM downloads Hadoop and creates the necessary configuration files, i.\,e. the \texttt{mapred-site.xml}, \texttt{core-site.xml}, \texttt{hdfs-site.xml}, \texttt{yarn-site.xml} and the slaves and master file containing the allocated nodes.
The node that is running the Agent are assigned to run the master daemons: the HDFS Namenode and the YARN Resource Manager.
After the configuration files are written, HDFS and YARN are started and meta-data about the cluster, i.\,e.\ the number of cores and memory, are provided to the scheduler.
They remain active until all the tasks are executed.
Before termination of the agent, the LRM stops the Hadoop and YARN daemons and removes the associated data files.
In Mode II (Hadoop on HPC), the LRM solely collects the cluster resource information.

The \emph{scheduler} is another extensible component of the RADICAL~-Pilot-Agent responsible for queueing compute units and assigning them to resources.
For YARN we created a special scheduler that utilizes updated cluster state information (e.\,g.\ the amount of available virtual cores, memory, queue information, application quotas etc.) obtained via the Resource Manager's REST API.
In contrast to other RADICAL~-Pilot schedulers, it specifically utilizes memory in addition to cores for assigning resource slots.

The \emph{Task Spawner} is responsible for managing and monitoring the execution of a Compute Unit.
The \emph{Launch Method} components encapsulates resource/launch-method specific operations, e.\,g.\ the usage of the \texttt{yarn} command line tool for submitting and monitoring applications.
After the launch of a Compute Unit, the Task Spawner periodically monitors its execution and updates its state in the shared MongoDB instance.
For YARN, the application log file is used for this purpose.

\begin{figure}[t]
    \centering
    \includegraphics[width=.95\textwidth]{figures/data_analytics_hpc/hpc_hadoop/yarn.pdf}
    \caption{RADICAL~-Pilot YARN Agent Application.}
%    \caption{\textbf{RADICAL~-Pilot YARN Agent Application: }
%        RADICAL~-Pilot provides a YARN application that manages the execution of Compute Units.
%        The application is initialized with parameters defined in the Compute Unit Description and started by the Task Spawner (step 1/2).
%        The Application Master requests resources from the Resource Manager and starts a container running the Compute Unit (step 3/4).}
    \label{fig:figures_yarn}
\end{figure}

\emph{RADICAL~-Pilot Application Master:}
A particular integration challenge is the multi-step resource allocation process imposed by YARN depicted in Figure~\ref{fig:figures_yarn}, which differs significantly from HPC schedulers.
The central component of a YARN application is the Application Master, which is responsible for negotiating resources with the YARN Resource Manager as well as for managing the execution of the application in the assigned resources.
The unit of allocation in YARN is a container (see~\cite{murthy2014apache}).
The YARN client (part of the YARN Launch Method) implements a YARN Application Master, which is the central instance for managing the resource requirements of the application.
RADICAL~-Pilot utilizes a managed application master that is run inside a YARN container.
Once the Application Master container is started, it is responsible for subsequent resource requests; in the next step it will request a YARN container meeting the resource requirements of the Compute Unit from the YARN's Resource Manager.
Once a set of container are allocated by YARN, the CU will be started inside these containers.
A wrapper script responsible for setting up a RADICAL~-Pilot environment, staging of the specified files and running the executable defined in the Compute Unit Description is used for this purpose.
Every Compute Unit is mapped to a YARN application consisting of an Application Master and a container of the size specified in the Compute Unit Description.
%In the future, we will further optimize the implementation by providing support for Application Master and container re-use.

\subsection{Spark Integration}
\label{sssec:rp_spark}
Spark offers multiple deployment modes: standalone, YARN and Mesos.
While it is possible to support Spark on top of YARN, this approach is associated with significant complexity and overhead as two instead of one framework need to be configured and bootstrapped.
Since RADICAL~-Pilot operates in user-space and single-user mode, no advantages with respect to using a multi-tenant YARN cluster environment exist.
Thus, we decided to support Spark via the standalone deployment mode.

Similar to the YARN integration, the necessary changes for Spark are confined to the RADICAL~-Pilot-Agent.
Similarly, the Local Resource Manager is responsible for initialization and deployment of the Apache Spark environment.
In the first step the LRM detects the number of cores, memory and nodes provided by the Resource Management System, verifies and downloads necessary dependencies (e.\,g.\ Java, Scala, and the necessary Spark binaries).
It then creates the necessary configuration files, e.\,g.\ \texttt{spark-env.sh}, \texttt{slaves} and \texttt{master} files, required for running a multi-node, standalone Spark cluster.

Finally, the LRM starts the Spark cluster using the previously generated configuration.
Similar to YARN, a Spark RADICAL~-Pilot-Agent scheduler is used for managing Spark resource slots and assigning CUs.
During the termination of the RADICAL~-Pilot-Agent, the LRM is shutting down the Spark cluster using Spark’s termination script, which stops both the master and the slave nodes.
Similarly, the Spark specific methods for launching and managing Compute Units on Spark are encapsulated in a Task Spawner and Launch Method component.

\section{Experiments and Evaluation}
\label{ssec:rph-exps}

To evaluate the RADICAL~-Pilot YARN and Spark extension, we conduct two experiments: in Section~\ref{sssec:startup_pilot_unit}, we analyze and compare RADICAL~-Pilot and RADICAL~-Pilot-YARN with respect to startup times of both the Pilot and the Compute Units.
We use the well-known K-Means algorithm to investigate the performance and runtime trade-offs of a typical data-intensive application.
Experiments are performed on two different XSEDE allocated machines: Wrangler~\cite{wrangler} and Stampede~\cite{stampede}.
On Stampede every node has 16 cores and 32 GB of memory; on Wrangler 48\,cores and 128\,GB of memory.
For our experiments we used RADICAL~-Pilot v0.45, Hadoop v2.6 and Spark v2.0.2.

\subsection{Pilot Startup and Compute Unit Submission}
\label{sssec:startup_pilot_unit}

\begin{figure}[t]
    \centering
    \includegraphics[width=0.85\textwidth]{figures/data_analytics_hpc/hpc_hadoop/pilot_unit_startup.pdf}
    \caption{RADICAL~-Pilot and RADICAL~-Pilot-YARN startup overheads.
        \label{fig:startup_yarn}}
%    \caption{\textbf{RADICAL~-Pilot and RADICAL~-Pilot-YARN startup overheads:}
%        The agent startup time is higher for YARN due to the overhead for spawning the YARN cluster.
%        The inset shows that the Compute Unit startup time (time between application submission to YARN and startup) is also significantly higher for YARN.
%        \label{fig:startup_yarn}}
\end{figure}

In Figure~\ref{fig:startup_yarn} we analyze the measured overheads when starting RADICAL~-Pilot and RADICAL~-Pilot-YARN, and when submitting Compute Units.
The agent startup time for RADICAL~-Pilot-YARN is defined as the time between RADICAL~-Pilot-Agent start and the processing of the first Compute Unit.
On Wrangler, we compare both Mode I (Hadoop on HPC) and Mode II (HPC on Hadoop).
For Mode I the startup time is higher compared to the normal RADICAL~-Pilot startup time and also compared to Mode II.
This can be explained by the necessary steps required for download, configuration and start of the YARN cluster.
For a single node YARN environment, the overhead for Mode I (Hadoop on HPC) is between 50-85\,sec depending upon the resource selected.
The startup times for Mode II on Wrangler -- using the dedicated Hadoop environment provided via the data portal -- are comparable to the normal RADICAL~-Pilot startup times as it is not necessary to spawn a Hadoop cluster.

The inset of Figure~\ref{fig:startup_yarn} shows the time taken to start Compute Units via RADICAL~-Pilot to a YARN cluster.
For each CU, resources have to be requested in two stages: first the application master container is allocated followed by the containers for the actual compute tasks.
For short-running jobs this represents a bottleneck.
%In the future, we will optimize this process by re-using the YARN application master and containers, which will reduce the startup time significantly.

In summary, while there are overheads associated with execution inside of YARN, we believe these are acceptable, in particular for long-running tasks.
The novel capabilities of executing HPC tasks and YARN tasks within the same application has significant benefits for which measured overheads are likely acceptable.

\subsection{K-Means}
\label{sssec:kmeans}
We compare the time to completion of the K-Means algorithm using two configurations: RADICAL~-Pilot on HPC and RADICAL~-Pilot in HPC/YARN mode. 

We use three different scenarios: 10,000 points and 5,000 clusters, 100,000 points / 500 clusters and 1,000,000 points / 50 clusters.
Each point belongs to a three dimensional space.
The compute requirements depend on the product of the number of points and number of clusters, thus it is constant for all three scenarios.
We use an optimized version of K-Means in which the sum and number of points are precomputed in the map phase, thus only these two attributes per cluster need to be shuffled.
The shuffle traffic depends on the number of clusters and decreases with the number of clusters.
For the purpose of this benchmark, we run two iterations of K-Means.

We utilize up to 3 nodes on Stampede and Wrangler.
On Stampede every node has 16 cores and 32 GB of memory; on Wrangler each node has 48\,cores and 128\,GB of memory.
Experiments are performed with the following configurations: 8~tasks on 1~node, 16~tasks on 2~nodes and 32~tasks on 3~nodes.
For RADICAL~-Pilot-YARN, we use Mode II (Hadoop on HPC): the YARN Resource Manager is deployed on the node running the RADICAL~-Pilot-Agent.

Figure~\ref{fig:experiments_kmeans_rpyarnkmeans} shows the results of executing K-Means over different scenarios and configurations.
For RADICAL~-Pilot-YARN the runtimes include the time required to download and start the YARN cluster on the allocated resources.

\begin{figure}[t]
    \centering
    \includegraphics[width=.95\textwidth]{figures/data_analytics_hpc/hpc_hadoop/kmeans.pdf}
    \caption{RADICAL~-Pilot and YARN-based K-Means time to completion on Stampede and Wrangler.}
%\caption{\textbf{RADICAL~-Pilot and YARN-based K-Means on Stampede and Wrangler:}
%        Across all configurations the performance of the YARN backend is in average 13\,\% better.
%        On Wrangler a significant better performance and scalability (higher speedups) were observed.}
    \label{fig:experiments_kmeans_rpyarnkmeans}
\end{figure}

Independent of the scenario, the runtimes decrease with the number of tasks.
In particular, in the 8 task scenarios the overhead of YARN is visible.
In particular for larger number of tasks, we observed on average 13\,\% shorter runtimes for RADICAL~-Pilot-YARN.
Also, RADICAL~-Pilot-YARN achieves better speedups, e.\,g., 3.2 for 32 tasks for the 1 million points scenario, which is significantly higher than the RADICAL~-Pilot speedup of 2.4 (both on Wrangler and compared to base case of 8 tasks).
One of the reason for this is that for RADICAL~-Pilot-YARN the local file system is used, while for RADICAL~-Pilot the Lustre filesystem is used.

For similar scenarios and task/resource configuration, the runtimes on Wrangler show a significant performance improvements over Stampede.
This is attributed to the better hardware (CPUs, memory).
In particular for RADICAL~-Pilot-YARN we observed on average higher speedups on Wrangler, indicating that we saturated the 32 GB of memory available on each Stampede node.

In summary, despite the overheads of RADICAL~-Pilot-YARN with respect to Pilot and Compute Unit startup time, we were able to observe performance improvements (on average 13\,\% better time to completion) mainly due to the better performance of the local disks.



\section{Modeling the task-parallel execution of data intensive workflows}
% !TEX root = main.tex
\label{ch:task-par}

Molecular Dynamics campaigns are significant consumer of computing cycles and
produce immense amounts of data, especially during simulation stages. Simulation
workflows execute up to $10^5$ $\mu sec$ MD simulations of a $O(100k)$ atoms
physical system, producing from $O(10)$ to $O(1000)$ GBs of
data~\cite{cheatham2015impact}. Increasingly, there is a need for analyses to be
integrated with simulations to drive the next stages of the workflow execution
at runtime~\cite{balasubramanian2016extasy}. Analyses can benefit from
task-level parallelism as they can be partitioned into independent units of
work.

Task-parallel applications involve partitioning a workflow into a set of
self-contained units of work. Based on the application, these tasks can be
independent, have no inter-task communication, or coupled with varying degrees
of data dependencies. Data-intensive applications exploit task parallelism for
data-parallel operations (e.g., \texttt{map}), but also require coupling, for
computing aggregates (\texttt{reduce}). Typically, a reduce operation includes
shuffling intermediate data from a set of nodes to node(s) where the reduce
executes.

Data analytics workflow engines exploit task-parallelism by partitioning the
data and generating stages of independent tasks. These stages then implement
data dependencies by shuffling, aggregating or coupling intermediate results.
The MapReduce~\cite{dean2004mapreduce} abstraction, along with its
implementations, popularized this method of processing.

Spark~\cite{zaharia2010spark} and Dask~\cite{rocklin2015dask} are two MapReduce
frameworks. Both provide abstractions which reason about data and are optimized
for parallel processing of large data volumes, interactive analytics and machine
learning. Their runtime engines can automatically partition data, generate
parallel tasks, and execute them on a cluster. In addition, Spark offers
in-memory capabilities allowing data caching data, making it suited for
interactive analytics and iterative machine learning algorithms. Dask also
provides a MapReduce API (Dask Bags). Furthermore, Dask's API is more versatile,
allowing custom workflows and parallel vector/matrix computations.

As task-parallel frameworks offer different abstractions and capabilities,
executing data analysis workflows in an efficient and scalable manner remains a
challenge. As data analysis applications can have different characteristics
(e.g., embarrassingly parallel or MapReduce), alternative abstractions and
capabilities may offer varying performance and efficiency. As a result, there is
a need to understand which framework is more suitable for performing a specific
type of analysis.

In this chapter, we investigate three task-parallel frameworks and their
suitability for implementing MD trajectory analysis algorithms. In addition to
Spark and Dask, we investigate RADICAL-Pilot (RP)~\cite{merzky2019using}. We utilize
MPI4py~\cite{dalcin2005mpi} to provide MPI equivalent implementations of the
algorithms. The task-parallel implementations performance and scalability
compared to MPI is the basis of our analysis. MD trajectories are time series of
atoms/particles positions and velocities, which are analyzed using different
statistical methods to infer certain properties, e.g., the relationship between
distinct trajectories, snapshots of a trajectory etc.

The chapter is organized as follows: ~\S\ref{sec:tp_related_work} discusses
several approaches to support scalable MD trajectory data analytics.
~\S\ref{sec:md_use_cases} describes the MD analysis algorithms investigated
and reviews different MD analysis frameworks with respect to their ability to
support scalable analytics of large volumes of MD trajectories.
~\S\\ref{sec:frameworks} describes RP, Spark and  Dask, the three
frameworks used for our performance comparison and evaluation.
~\S\ref{sec:impl_exp} provides a description of the implementation of the
MD algorithms on top of those three frameworks, as well as a performance
evaluation and a discussion of our findings. In
~\S\ref{sec:task_sel_model}, we provide a conceptual analysis that allows
application developers to select a framework according to their requirements.
Finally, ~\S\ref{sec:tp_concl} closes this chapter with our conclusions.

%\mtnote{In Chapter 2 you use \S for section, here you use "s|Section \#"}

\section{Related Work}
\label{sec:tp_related_work}

Until recently, MD analysis algorithms were executed serially and
parallelization was not straightforward. During the last years, several
frameworks emerged providing parallel algorithms for analyzing MD trajectories.
Some of those frameworks are CPPTraj~\cite{roe2013ptraj,roe2018parallelization},
HiMach~\cite{tiankai2008scalable}, PMDA~\cite{fan2019pmda}, Pteros
2.0~\cite{yesylevskyy2015pteros} and nMoldyn-3~\cite{hinsen2012nmoldyn}.

Several techniques are used for parallelizing MD analysis algorithms.
CPPTraj~\cite{roe2018parallelization} utilizes MPI and OpenMP to execute
large-scale analysis on HPC. OpenMP is also utilized by
Pteros~\cite{yesylevskyy2015pteros} to parallelize the compute intensive parts
of the analysis. HiMach~~\cite{tiankai2008scalable} extends Google's MapReduce
and defines Map and Reduce methods. nMoldyn-3~\cite{hinsen2012nmoldyn}
parallelizes the execution through a Master/Worker architecture. The master
defines analysis tasks which are then executed by a set of worker processes.

%HiMach~\cite{tiankai2008scalable} was developed by D. E. Shaw Research group to
%provide a parallel analysis framework for MD simulations, and extends Google's
%MapReduce. HiMach API defines trajectories, does per frame data acquisition
%(Map) and cross-frame analysis (Reduce). HiMach's runtime is responsible to
%parallelize and distribute Map and Reduce phases to resources. Data transfers
%are done through a communication protocol created specifically for HiMach.

%Pteros-2.0~\cite{yesylevskyy2015pteros} is a open-source library that is used
%for modeling and analyzing MD trajectories, providing a plugin for each
%supported algorithm. The execution is done by a user defined driver
%application, which setups trajectory  I/O and frame dispatch for analysis. It
%offers a C++ and Python API. Pteros 2.0 parallelizes computational intensive
%algorithms via OpenMP and Multithreading. As a result, it is bounded to execute
%on a single node, making any analysis execution highly dependent on memory
%size. Through RP, Spark and Dask, we avoided recompiling every time
%there is a change to the underlying resource, ensuring the application's
%execution.

%MDTraj~\cite{mcgibbon2015mdtraj} is a Python package for analyzing MD
%trajectories. It links MD data and Python statistical and visualization
%software. MDTraj proposes parallelizing the execution by using the parallel
%package of IPython as a wrapper along with an out-of-core trajectory reading
%method. Our approach allows data parallelization on any level of the execution,
%not only in data read.

%nMoldyn-3~\cite{hinsen2012nmoldyn} parallelizes the execution through a Master
%Worker architecture. The master defines analysis tasks, submits them to a task
%manager, which then are executed by the worker process. In addition, it
%provides adaptability, allowing on-the-fly addition of resources, and execution
%fault tolerance when worker processes disconnect.

A common denominator of most approaches is that they do not use general purpose
frameworks for parallelizing the execution. Although they are optimized to get
as much performance as possible from the environments they are developed for,
their portability is limited. For example, CPPTraj~\cite{roe2018parallelization}
and Pteros~\cite{yesylevskyy2015pteros} are highly dependent on the low level
libraries of the resource they use., while HiMach~\cite{tiankai2008scalable} is
build specifically for the Antons Supercomputer.

In contrast, our analysis focuses on frameworks that offer more general purpose approaches to the
parallelization of MD analysis algorithms.
%Specifically, PMDA utilizes Dask to parallelize MD Trajectory analysis.
These frameworks provide higher-level abstractions that facilitates the
integration with other data analysis methods. In addition, resource acquisition
and management is done transparently.

\section{Molecular Dynamics (MD) Data Analysis}
\label{sec:md_use_cases}

Root Mean Square Deviation (RMSD), Pairwise Distances (PD), and
Sub-setting~\cite{mura2014biomolecules} are algorithms commonly used to analyze
MD trajectories. Path Similarity Analysis (PSA)~\cite{seyler2015path} and
Leaflet Identification~\cite{michaud2011mdanalysis} are two more advanced
algorithms. All these methods read and process a physical system generated via
multiple simulations, reducing data to either a single number or a matrix. RMSD
identifies the deviation of atom positions among frames, while PD and PSA
calculate distances between atoms or trajectories based on different metrics.
Sub-setting methods are used instead to isolate parts of interest of a MD
simulation, and Leaflet Identification provides information about groups of
lipids by identifying the lipid leaflets in a lipid bilayer.

We discuss two of those methods---a PSA algorithm that uses the Hausdorff
distance and a Leaflet identification algorithm---implementated in
MDAnalysis~\cite{michaud2011mdanalysis,gowers2016mdanalysis}.
Section~\ref{ssec:mda} discusses the reasons for selecting these two algorithms
in more detail.
%In addition, we implemented the PSA algorithm using CPPTraj~\cite{roe2013ptraj}.
%Furthermore, we explore the applications' Ogres Facets and Views~\cite{fox2014towards}, which provide a more systematic characterization.

%Big Data Ogres~\cite{fox2014towards} are organized into four classes, called \emph{views}.
%The possible features of a view are called \emph{facets}.
%A combination of facets from all views defines an Ogre.
%The Views are:
%\begin{inparaenum}[1)]
%    \item execution - describes aspects, such as I/O, memory, compute ratios, whether computations are iterative, and the 5 V's of Big Data (Volume, Velocity, Value, Variety and Veracity),
%    \item data source \& style - discusses input data collection, storage and access,
%    \item processing - describes algorithms and kernels used for computation, and
%    \item problem architecture - describes the application architecture.
%\end{inparaenum}


\subsection{Applications and Algorithms}
\label{ssec:mda}

MDAnalysis is a Python library~\cite{michaud2011mdanalysis,gowers2016mdanalysis}
that provides a comprehensive environment for filtering, transforming and
analyzing MD trajectories in all commonly used file formats. MDAnalysis provides
a common object-oriented API to trajectory data and leverages existing libraries
of the scientific Python software stack, such as NumPy~\cite{numpy} and
Scipy~\cite{scipy}.

\subsubsection*{Path Similarity Analysis (PSA): Hausdorff Distance}

Path Similarity Analysis (PSA)~\cite{seyler2015path} quantifies the
similarity between trajectories, considering their full atomic detail. The
basic idea is to compute pair-wise distances (e.g., using the Hausdorff
metric~\cite{huttenlocher1993comparing}) between members of an ensemble of
trajectories, and cluster the trajectories based on their distance matrix. Each
trajectory is represented as a two dimensional array: The first dimension
corresponds to time frames of the trajectory; the second to the $N$ atom
positions in a 3-dimensional space.

\begin{algorithm}[t]
    \scriptsize
    \caption{Path Similarity Algorithm: Hausdorff Distance}
    \label{alg:hausdorff}
    \begin{algorithmic}[1]
        \Procedure{HausdorffDistance}{$T_1$,$T_2$}\Comment{$T_1$ and $T_2$ are a set of
            3D points}
        \State \texttt{List $D_1$,$D_2$}
        \For{$\forall frame_1$ in $T_1$}
        \For{$\forall frame_2$ in $T_2$}
        \State \texttt{Append in $D_1$ $d_{RMS}$($frame_1$, $frame_2$)}
        \EndFor
        \State \texttt{$D_{t_1}$ append $min(D_1)$}
        \EndFor
        \For{$\forall frame_2$ in $T_2$}
        \For{$\forall frame_1$ in $T_1$}
        \State \texttt{Append in $D_2$ $d_{RMS}$($frame_2$, $frame_1$)}
        \EndFor
        \State\texttt{$D_{t_2}$ append $min(D_2)$}
        \EndFor
        \State \textbf{return} $max\Big(max(D_{t_1}),max(D_{t_2})\Big)$
        \EndProcedure
        \\
        \Procedure{PSA}{$Traj$}\Comment{$Traj$ is a set of trajectories}
        \For{$\forall T_1$ in $Traj$}
        \For{$\forall T_2$ in $Traj$}
        \State \texttt{ $D_{( T_1,T_2 )}$=HausdorffDistance$\Big( T_1,T_2 \Big)$}
        \EndFor
        \EndFor
        \State \Return $D$
        \EndProcedure
    \end{algorithmic}
\end{algorithm}

Algorithm~\ref{alg:hausdorff} describes PSA with the Hausdorff metric over
multiple trajectories. We apply a 2-dimensional data partitioning over the
output matrix to parallelize, shown in algorithm~\ref{alg:partition}. Our
Hausdorff metric calculation is based on a naive algorithm.
%Recently, an algorithm was introduced that uses early break to speedup execution~\cite{taha2015efficient}, although we are not aware of a parallel implementation of this algorithm.

\begin{algorithm}[t]
    \scriptsize
    \caption{Two Dimensional Partitioning}
    \label{alg:partition}
    \begin{algorithmic}[1]
        \State Initially, there are $N^2$ distances, where $N$ is the number of trajectories.
        Each distance defines a computation task.
        \State Map the initial set to a smaller set with $k=N/n_1$ elements, where $n_1$ is a divisor of $N$, by grouping $n_1$ by $n_1$ elements together.
        \State Execute over the new set with $k^2$ tasks.
        Each task is the comparisons between $n_1$ and $n_1$  elements of the initial set.
        They are executed serially.
    \end{algorithmic}
\end{algorithm}

The algorithm is embarrassingly parallel, has complexity $O(n^2)$ and its input
data volume is medium to large while the output is small.
%Specific execution environments, such as HPC nodes, and Python arithmetic
%libraries, e.g., NumPy, are used (execution view). Input data are produced by
%HPC simulations, and are stored on HPC storage systems, such as parallel
%filesystem like Lustre (data source \& style view).
Embarrassingly parallel algorithms are good candidates for task parallelization.
The dataset is partitioned and each partition is assigned to a task. Given
enough resources, task can execute concurrently as a bag of tasks, using a task
management API or a map-only application in a MapReduce-style API. Spark, Dask
and RP support the execution of bag of tasks, with RP and
Dask offering specific abstractions (as discussed in~\S\ref{sec:frameworks}).

\subsubsection*{Leaflet Finder}

Algorithm~\ref{alg:leafletfinder} describes the Leaflet Finder (LF) algorithm as
presented in Ref.~\cite{michaud2011mdanalysis}. LF assigns particles to one of
two curved but locally approximately parallel sheets, provided that the
inter-particle distance is smaller than the distance between sheets. In
biomolecular simulation data analysis of lipid membranes, consisting of a double
layer of lipid molecules, LF identifies the lipids of the outer and inner
leaflets (sheets). The algorithm consists of two stages:
\begin{inparaenum}[a)]
    \item construction of a graph connecting particles based on threshold
    distance (cutoff); and
    \item computing the connected components of the graph, determining the
    lipids located on the outer and inner leaflets.
\end{inparaenum}

\begin{algorithm}[t]
    \scriptsize
    \caption{Leaflet Finder Algorithm}
    \label{alg:leafletfinder}
    \begin{algorithmic}[1]
        \Procedure{LeafletFinder}{$Atoms,Cutoff$}
        \Comment{$Atoms$ is a set of 3D points that represent the position of atoms in space. $Cutoff$ is an Integer Number}
        \State \texttt{Graph G =$(V=Atoms,E=\emptyset)$}
        \For{$\forall atom$ in $Atoms$}
        \State \texttt{$N = [a\in V: d(a,atom)\le Cutoff]$}
        \State \texttt{Add edges $[(atoms,a): a \in N]$ in G}
        \EndFor
        \State \texttt{C = ConnectedComponents(G)}
        \State \Return C
        \EndProcedure
    \end{algorithmic}
\end{algorithm}

The application stages have different complexities. The first stage identifies
neighboring atoms. There are two alternative implementations:
\begin{inparaenum}[i)]
    \item computing the distance between all atoms ($O(n^2)$);
    \item utilizing a tree-based nearest neighbor (Construction: $O(n\log n)$,
    Query: $O(\log n)$).
\end{inparaenum}
In both alternatives, the input data volume is medium size and the output is
smaller than the input. The complexity of connected components is: $O(|V|+|E|)$
($V$: Vertices, $E$: Edges), i.e., it greatly depends on the characteristics of
the graph.
%The application typically uses HPC nodes as the execution environment, and
%NumPy arrays. It uses matrices to represent the physical system and the
%distance matrix. The output data representation is a graph. Leaflet Finder can
%be efficiently implemented using the MapReduce abstraction. It uses graph
%algorithms and linear algebra kernels (processing view facets). The data source
%\& style view facets are the same as the PSA algorithm.

LF is more complex than PSA as it requires two stages. It is possible to
implement LF with a simple task-management API, although the MapReduce
programming model allows for a more efficient implementation with a \texttt{map}
for computing and filtering distances and a \texttt{reduce} for finding the
components. The shuffling required between map and reduce is medium as the
number of edges is a fraction of the input data. Spark and Dask natively support
MapReduce and implementing LF is relatively straightforward. RP, on
the other hand, does not natively support MapReduce. As a result, the
application developer has to define the data shuffling between the \emph{map}
and the \emph{reduce} phases of the algorithm.

\section{Task-Parallel Frameworks: Spark, Dask and RADICAL-Pilot}
\label{sec:frameworks}

The landscape of frameworks for data-intensive applications is
manifold~\cite{jha2014tale,kamburugamuve2017anatomy} and has been extensively
studied in the context of scientific~\cite{jha2017introducing} applications. In
this section, we discuss the suitability of task-parallel frameworks such as
Spark~\cite{zaharia2010spark}, Dask~\cite{rocklin2015dask} and
RP~\cite{merzky2019using} for MD data analytics.

\subsection{Background}

Spark and Dask provide APIs, caching and other capabilities that are critical to
develop analytics applications. Spark is considered the standard solution for
iterative data-parallel applications, while Dask is quickly gaining support by
the scientific community, since it offers a sought-after Python environment.
RP also offers a Python-only programming API, supporting task-level
parallelism on HPC resources. In this way, RP adds parallelization
capabilities to MPI-based applications, enabling the concurrent and sequential
execution of bag of MPI tasks on HPC resources.

As described in~\cite{jha2014tale}, these frameworks typically comprise of
distinct layers, e.g., cluster scheduler access, framework-level scheduling, and
higher-level abstractions. Various higher-level abstractions can be provided on
top of low-level resource management capabilities, e.g., MapReduce-inspired
APIs. These capabilities provide the foundation for analytics abstractions, such
as Dataframes, Datasets and Arrays.
Figure~\ref{fig:figures_bigdata_framework_stack} visualizes the components of
RP, Spark and Dask.

\begin{figure}[ht]
    \centering
    \includegraphics[width=.95\textwidth]{figures/data_analytics_hpc/task_par/bigdata_framework_stack.pdf}
    \caption{Architecture of RADICAL-Pilot, Spark and Dask.}
    %\caption{\textbf{Architecture of RADICAL-Pilot, Spark and Dask:}
    %The frameworks share common architectural components for managing cluster resource, and tasks.
    %Spark, Dask offer several high-level abstractions inspired by MapReduce.}
    \label{fig:figures_bigdata_framework_stack}
\end{figure}

\subsubsection*{RADICAL-Pilot}

%RADICAL-Pilot~\cite{merzky2019using} is a Pilot system that implements the
%pilot paradigm as outlined in Ref.~\cite{turilli2018comprehensive}.
%RADICAL-Pilot (RP) is implemented in Python and provides a well defined API and
%usage modes. Although RP is vehicle for research in scalable computing, it also
%supports production grade science. Currently, it is being used by applications
%drawn from diverse domains, ranging from earth and biomolecular sciences to
%high-energy physics. RP can be used as a runtime system by workflow or workload
%management
%systems~\cite{turilli2019middleware,treikalis2016repex,balasubramanian2018harnessing,dakka2018high,turilli2017evaluating}.
%In 2017, RP was used to support more than 100M core-hours on US DOE, NSF
%resources (BlueWaters and XSEDE), and European supercomputers (Archer and
%SuperMUC).

As discussed in~\S\ref{ch:pilot-data-hadoop}, RP allows concurrent
task execution on HPC resources. The user defines a set of Compute-Units (CU)
which are submitted to RP. RP schedules these CUs for
execution on the acquired resources and uses the existing environment of the
resource to execute tasks. Any data communication between tasks is done via an
underlying shared filesystem, e.g., Lustre. Task execution coordination and
communication is done through a database (MongoDB).

%RP's learning curve can be quite steep at the beginning, at least
%until the user becomes familiar with the concept and usability of Pilots and
%CUs. Once the user is comfortable with RP's API, she can easily
%develop new algorithms.

\subsubsection*{Spark}

Spark~\cite{zaharia2010spark} extends MapReduce~\cite{dean2004mapreduce},
providing a rich set of operations on top of the Resilient Distributed Dataset
(RDD) abstraction~\cite{zaharia2012resilient}. RDDs are cached in-memory, making
Spark well suited for iterative applications that need to cache a set of data
across multiple stages. PySpark provides a Python API to Spark.

A Spark workflow consists of multiple stages. Each stage is a set of independent
parallel tasks (e.g., \texttt{map}) and an action (e.g., \texttt{reduce}).
Spark's Scheduler translates the workflow specified via RDD transformations and
actions to an execution plan. Its distributed execution engine handles the
low-level details of task execution, which is triggered by actions. Spark can
read data from different sources, such as HDFS, blob storage, parallel and local
filesystems. While Spark caches loaded data in memory, it offloads to disk when
there is not enough free memory on a node. Persisted RDDs remain in memory,
unless specified to use the disk either complementary or as a single target. In
addition, Spark writes data that are used in a shuffle to disk. As a result, it
allows quick access to those data when transmitted to an executor.
%Finally, Spark provides a set of actions that write text files, Hadoop sequence files or object files to local filesystems, HDFS or any filesystem that supports Hadoop.
%In addition, Spark supports higher-level data abstractions for processing structured data, such as dataframes, Spark-SQL, datasets, and data streams.

\subsubsection*{Dask}

Dask~\cite{rocklin2015dask} provides a Python-based parallel computing library,
which is designed to parallelize native Python code written for NumPy and
Pandas. In contrast to Spark, Dask also provides a lower-level task API
(\texttt{delayed} API) that allows users to construct arbitrary workflow graphs.
Being written in Python, it does not require to translate data types from one
language to another like PySpark, which moves data between Python's interpreter
and Java/Scala.

In addition to the low-level task API, Dask offers three higher-level
abstractions: Bags, Arrays and Dataframes. Dask Arrays are a collection of NumPy
arrays organized as a grid. Dask Bags are similar to Spark RDDs and are used to
analyze semi-structured data, like JSON files. Dask Dataframes are distributed
collections of Pandas dataframes that can be analyzed in parallel.

Dask also offers three schedulers: multithreading, multiprocessing and
distributed. The multithreaded and multiprocessing schedulers can be used only
on a single node and the parallel execution is done via threads and processes
respectively. The distributed scheduler creates a cluster with a scheduling
process and multiple worker processes. A client process creates and communicates
a workflow as a direct acyclic graph to the scheduler. Finally, the scheduler
assigns tasks to workers.

%Dask's learning curve cannot be considered steep. Its API is well defined and
%documented. In addition, familiarity with Spark or MapReduce helps to minimize
%the learning curve even further. As a result, implementing MD analysis
%algorithms on Dask did not require significant engineering time. In addition,
%setting up a Dask cluster on a set of resources was relatively straightforward,
%since it provides all the binaries, e.g. \texttt(dask-ssh).

%\subsubsection*{RP} RP~\cite{merzky2019using} is a Pilot
%system that implements the pilot paradigm as outlined in
%Ref.~\cite{turilli2018comprehensive}. RP (RP) is implemented in
%Python and provides a well defined API and usage modes. Although RP is vehicle
%for research in scalable computing, it also supports production grade science.
%Currently, it is being used by applications drawn from diverse domains, ranging
%from earth and biomolecular sciences to high-energy physics. RP can be used as
%a runtime system by workflow or workload management
%systems~\cite{turilli2019middleware,treikalis2016repex,balasubramanian2018harnessing,dakka2018high,turilli2017evaluating}.
%In 2017, RP was used to support more than 100M core-hours on US DOE, NSF
%resources (BlueWaters and XSEDE), and European supercomputers (Archer and
%SuperMUC).

%RP allows concurrent task execution on HPC resources. The user
%defines a set of Compute~-Units (CU)~- the abstraction that defines a task
%along with its dependencies - which are submitted to RP.
%RP schedules these CUs to be executed under the acquired resources.
%It uses the existing environment of the resource to execute tasks. Any data
%communication between tasks is done via an underlying shared filesystem, e.g.,
%Lustre. Task execution coordination and communication is done through a
%database (MongoDB).

%RP's learning curve can be quite steep at the beginning, at least
%until the user becomes familiar with the concept and usability of Pilots and
%CUs. Once the user is comfortable with RP's API, she can easily
%develop new algorithms.

\subsection{Comparison}
\begin{table}[t]
    \scriptsize
    \centering
    \begin{tabular}{@{}p{2.75cm}|p{3.25cm}p{3.25cm}p{3.25cm}@{}}
        \toprule
        &\textbf{RADICAL-Pilot} &
        \textbf{Spark} &
        \textbf{Dask} \\
        \midrule
        % row 1
        Languages &
        Python &
        Java, Scala, Python, R &
        Python\\
        % Row 2
        Task &
        Task &
        Map-Task &
        Delayed\\
        % row 3
        Abstraction &
        &
        & \\
        % row 4
        Functional Abstraction  &
        - &
        RDD API &
        Bag\\
        % row 5
        Higher-Level Abstractions &
        Pipelines~\cite{balasubramanian2018harnessing}, Replicas~\cite{dakka2018concurrent} &
        Dataframe, ML Pipeline, MLlib~\cite{meng2016mllib} &
        Dataframe, Arrays for block computations\\
        % row 6
        Resource Management &
        Pilot-Job &
        Spark Execution Engines &
        Dask Distributed Scheduler\\
        % row 7
        Scheduler    &
        Individual Tasks &
        Stage-oriented DAG &
        DAG\\
        % row 8
        Shuffle      &
        -       &
        hash/sort-based shuffle &
        hash/sort-based shuffle\\
        % row 8
        Limitations &
        no shuffle, filesystem-based communication  &
        high overheads for Python tasks (serialization)   &
        Dask Array can not deal with dynamic output shapes\\
        \bottomrule
    \end{tabular}
    \caption{Task-parallel Frameworks Comparison.\label{tab:frameworks}}
    %\caption{\textbf{Task-parallel Frameworks Comparison:} Dask and Spark are designed for data-related task, while RADICAL-Pilot focuses on compute-intensive tasks.\label{tab:frameworks}}
\end{table}

Table~\ref{tab:frameworks} summarizes the properties of Spark, Dask and
RP with respect to the abstractions and runtime systems provided to
create and execute parallel data applications.

\subsubsection*{API and Abstractions}

RP provides a low-level API for executing tasks onto resources. While
this API can be used to implement high-level capabilities, e.g.,
MapReduce~\cite{mantha2012pilot}, they are not provided out-of-the box. Both
Spark and Dask provide such capabilities. Dask's API is generally lower level
than Spark's, i.e., it allows specifying arbitrary task graphs. Although data
partition size is automatically decided, in many cases it is necessary to tune
parallelism by specifying the number of partitions.

Another important aspect is the availability of high-level abstractions.
High-level abstractions for RP, such as 
Pipelines~\cite{balasubramanian2018harnessing} and 
Replicas~\cite{dakka2018concurrent}, are designed for compute-oriented 
tasks. Contrary, Spark and Dask already offer a set of
high-level data-oriented abstractions, such as Dataframes. 

%\mtnote{I am not sure
%those are a valid comparisons: high-level abstraction Vs. tools; dataframes Vs.
%workflows. I would eliminate tools, using only abstractions. I would then
%compare DataFrame to Pipelines, possibly adding Replicas and a reference to
%HTBAC.}

\subsubsection*{Scheduling}

Both Spark and Dask create a Direct Acyclic Graph (DAG) based on operations over
data, which is then executed using their execution engine. Spark jobs are
separated into stages. When a stage is completed, the scheduler executes the
next stage.

Dask's DAGs are represented by a tree where each node is a task. Leaf tasks do
not depend on other task for execution. Dask tasks are executed when their
dependencies are satisfied, starting from leaf tasks. When a task is reached
with unsatisfied dependencies, the scheduler executes the dependent task first.
Dask's scheduler does not rely on synchronization points that Spark's
stage-oriented scheduler introduces. RP does not provide a DAG-based
API and requires the user to manage the execution order and synchronization
among tasks at application level instead of runtime level (i.e., RP).

\subsection{Frameworks Evaluation}
\label{sec:framework_eval}

As data-parallelism often involves a large number of short-running tasks, task
throughput is a critical metric to assess the three frameworks we consider. To
evaluate the throughput of those frameworks, we use zero workload tasks
(\texttt{/bin/hostname}). We submit an increasing number of such tasks to
RP, Spark and Dask and measure the execution time on a single node.

For RP, all tasks were submitted simultaneously. RP's
backend database was running on the same node to avoid communication
latencies. For Spark, we created an RDD with as many partitions as the number of
tasks as each partition is mapped to a task by Spark. For Dask, we created tasks
using \texttt{delayed} functions that were executed by the Distributed
scheduler. We used TACC Wrangler and SDSC Comet for this experiment. SDSC Comet
is a 2.7 petaFLOSP cluster with 24 Haswell cores/node and 128\,GB memory/node
(6,400 nodes). TACC Wrangler has 24 Haswell hyper-threading enabled cores/node
and 128\,GB memory/node (120 nodes).

\begin{figure}[t]
    \centering
    \includegraphics[width=.75\textwidth]{figures/data_analytics_hpc/task_par/dask_spark_rp_wrangler.pdf}
    \caption{Total time to completion and task throughput by framework on a
    single node for an increasing number of tasks on Wrangler.}
    \label{fig:dask_spark_rp_wrangler}
\end{figure}

Figure~\ref{fig:dask_spark_rp_wrangler} shows the total time to completion and
task throughput. Dask needed the least time to schedule and execute the assigned
tasks, followed by Spark and RP. Dask and Spark quickly reach their
maximum throughput, which is sustained as the number of tasks increased.
RP showed the worst throughput and scalability, mainly due to some
architectural limitations. Specifically, RP relies on a MongoDB
database to communicate between Client and Agent, as well as several components
that allow RP to move data and introduce delays in the execution of
the tasks. As a result, we were not able to scale RP to 32k or more
tasks.

\begin{figure}[t]
    \centering
    \includegraphics[width=.75\textwidth]{figures/data_analytics_hpc/task_par/daskVSsparkVSRpThroughput.pdf}
    \caption{Task throughput by framewrork for $100k$ tasks on different number of nodes.}
    \label{fig:RP_Dask_Spark_throughput}
\end{figure}

Figure~\ref{fig:RP_Dask_Spark_throughput} illustrates task throughput when
scaling to multiple nodes, measured by submitting $100k$ tasks. Dask's
throughput on both resources increases almost linearly to the number of nodes.
Spark's throughput is an order of magnitude lower than Dask's. RP's
throughput plateaus below $100 task/sec$. Wrangler and Comet show a comparable
performance, with Comet slightly outperforming Wrangler.

%\subsubsection*{Suitability for MDAnalysis Algorithms} Trajectory analysis
%methods are often embarrassingly parallel. So, they are ideally suited for task
%management and MapReduce APIs. PSA-like methods typically require a single pass
%over the data and return a set of values that correspond to a relationship
%between frames or trajectories. They can be expressed as a bag of tasks using a
%task management API or a map-only application in a MapReduce-style API.

%Leaflet Finder is more complex and requires two stages: \begin{inparaenum}[a)]
%\item the edge discovery stage, and \item the connected components stage.
%\end{inparaenum} It is possible to implement Leaflet Finder with a simple
%task-management API, although the MapReduce programming model allows more
%efficient implementation with a \texttt{map} for computing and filtering
%distances and a \texttt{reduce} for finding the components. The shuffling
%required between map and reduce is medium as the number of edges is a fraction
%of the input data.

\section{Task-Parallel MD Trajectory Data Analysis: Implementation \& Characterization}
\label{sec:impl_exp}

In this section, we characterize and evaluate the performance of two algorithms
from MDAnalysis---PSA and LF---using different real-world datasets,
when implemented with RP, Spark and Dask. We compare the performance
of these three task-parallel frameworks with an equivalent implementations of
the algorithms using MPI4py. We investigate:
\begin{inparaenum}[1)]
    \item which capabilities and abstractions of the frameworks are needed to
    efficiently express these algorithms;
    \item what architectural approaches can be used to implement these
    algorithms with these frameworks; and
    \item the performance trade-offs of these frameworks.
\end{inparaenum}

The experiments were executed on SDSC Comet and TACC Wrangler. Experiments were
carried using RP and Pilot-Spark (as discussed
on~\S\ref{ch:pilot-data-hadoop}). We utilize a set of custom scripts to start
the Dask cluster. We used RP 0.46.3, Spark 2.2.0, Dask 0.14.1 and
Distributed 1.16.3.
%The data presented are means over multiple runs; error bars represent the
%standard deviation of the sample.
We employed up to $10$ nodes on Comet and Wrangler.

\subsection{Path Similarity Analysis (PSA): Hausdorff Distance}
\label{sec:psa}

The PSA algorithm is embarrassingly parallel and can be implemented using simple
task-level parallelism or a map-only MapReduce application. The input data,
i.e., a set of trajectory files, is equally distributed over the cores,
generating one task per core. Each task reads its respective input files in
parallel, executes and writes the result to a file.

For RP, we define a Compute-Unit for each task and execute them using
a Pilot-Job. For Spark, we create an RDD with one data partition per task, which
are executed in a \texttt{map} function. In Dask, tasks are defined as
\texttt{delayed} functions and in MPI, each task is executed by an MPI process.
The dataset used for the experiments consists of three atom count trajectories:
\begin{inparaenum}[1)]
    \item small ($3341$ atoms/frame);
    \item medium ($6682$ atoms/frame); and
    \item large ($13364$ atoms/frame).
\end{inparaenum}
We used $102$ frames, and $128$ and $256$ trajectories of each size.

Figure~\ref{fig:HausdorffWrangler} shows the runtime for 128 and 256
trajectories on Wrangler, while Figure~\ref{fig:comet_wrangler_haus} compares
the execution times on Comet and Wrangler for $128$ large trajectories. The
three frameworks show similar performance in terms of runtime on both systems,
as well as with MPI4py. Wrangler gives smaller speedup than Comet
despite using the same number of cores. Wrangler uses hyperthreading and, as a
result, multiple tasks share the same CPU, resulting in lower speedup than
Comet.

\begin{figure}[t]
    \centering
    \includegraphics[width=0.85\textwidth]{figures/data_analytics_hpc/task_par/HausdorffSingleFig.pdf}
    \caption{Time to completion of PSA on Wrangler using RADICAL-Pilot, Spark
    and Dask over different number of cores, trajectory sizes, and number of
    trajectories.}
    \label{fig:HausdorffWrangler}
\end{figure}

\begin{figure}[t]
    \centering
    \includegraphics[width=.85\textwidth]{figures/data_analytics_hpc/task_par/comet_wrangler_haus.pdf}
    \caption{Time to completion and speedup of PSA on Comet and Wrangler for 128 large trajectories.}
    \label{fig:comet_wrangler_haus}
\end{figure}

MPI4py, RP, Spark and Dask have similar performance when used to
execute embarrassingly parallel algorithms. All frameworks achieved similar
speedups as the number of cores increased, scaling by a factor of 6 from 16 to
256 cores, which are lower than MPI4py. Although the frameworks' overheads are
comparably low in relation to the overall runtime, overheads were significant
enough to impact the frameworks' speedup. Note that RP's large
deviation is due to sensitivity to communication delays with the database.

In summary, for embarrassingly parallel algorithms all three frameworks provide
appropriate abstractions and runtime performance compared to MPI. Beyond
performance considerations, aspects such as programmability and
integrate-ability are more important considerations when comparing the three
frameworks. Both RP and Dask are native Python frameworks, making the
integration with other Python tools easier and more efficient than with
frameworks which are based on other languages like Spark.

%\begin{figure}[ht]
%    \centering
%    \includegraphics[width=.95\textwidth]{figures/data_analytics_hpc/task_par/cpptrajHausdorff.pdf}
%    \caption{\label{fig:cpptraj_resutls}\textbf{Hausdorff Distance using CPPTraj:}
%    Runtimes and Speedup over different number of cores.}
%\end{figure}

%CPPTraj~\cite{roe2018parallelization} provides an optimized C++ implementation of the 2D-RMSD, which is Algorithm~\ref{alg:hausdorff} with no $\min-\max$ operations.
%The 2D-RMSD between trajectories was executed in parallel.
%The results were gathered and the Hausdorff distance was calculated.
%CPPTraj~\cite{roe2018parallelization} was compiled with GNU C++ compiler and no optimizations, and with Intel's compiler O3 optimization enabled.
%An experiment was run with 20-core Haswell nodes and 128 small trajectories; number of cores ranging from 1 up to 240.
%Figure~\ref{fig:cpptraj_resutls} shows the runtimes and speedup.
%MPI C++ provides lower execution times.
%However, we are interested in scalable solutions, that may offer worse performance in absolute numbers, but allows easier integration, i.e., less lines of code, and/or less engineering time.

\subsection{Leaflet Finder (LF)}
\label{sec:leaflet}

\begin{table*}[t]
    \scriptsize
    \centering
    \begin{tabular}{@{}p{2cm}|p{2.8cm}p{2.8cm}p{2.8cm}p{2.8cm}@{}}
        \toprule
        &
        \textbf{Broadcast and 1-D} (Approach 1) &
        \textbf{Task API and 2-D} (Approach 2) &
        \textbf{Parallel Connected Components} (Approach 3) &
        \textbf{Tree-Search} (Approach 4)\\
        \midrule
        % row 1
        Data Partitioning  &
        1D  &
        2D &
        2D &
        2D\\
        % row 2
        Map &
        Edge Discovery via Pairwise Distance &
        Edge Discovery via Pairwise Distance &
        Edge Discovery via Pairwise Distance and Partial Connected Components &
        Edge Discovery via Tree-based Algorithm and Partial Connected Components\\
        % row 3
        Shuffle &
        Edge List ($O(E)$) &
        Edge List ($O(E)$) &
        Partial Connected components ($O(n)$) &
        Partial Connected components ($O(n)$)\\
        % row 4
        Reduce   &
        Connect Components  &
        Connected Components &
        Joined Connected Components &
        Joined Connected Components\\
        \bottomrule
    \end{tabular}
    \caption{MapReduce Operations used by Leaflet Finder\label{tab:app_operators}}
\end{table*}

\begin{figure*}[t]
    \centering
    \includegraphics[width=.95\textwidth]{figures/data_analytics_hpc/task_par/lf_approaches.pdf}
    \caption{Architectural approaches for implementing the Leaflet finder algorithm\label{fig:lf_approaches}}
\end{figure*}

We developed five approaches to implement the LF algorithm using RP,
Spark, Dask, and MPI4py (see Fig~\ref{fig:lf_approaches} and
Table~\ref{tab:app_operators}):
\begin{enumerate}[1)]
    \item \textbf{Broadcast and 1-D Partitioning:} The physical system is
    broadcasted and partitioned through a data abstraction. Use of RDD API
    (broadcast), Dask Bag API (scatter), and MPI Bcast to distribute data to all
    nodes. A \texttt{map} function calculates the edge list using \texttt{cdist}
    from SciPy~\cite{scipy}, realized as a loop for MPI. The list is collected
    to the master process (gathered to rank 0) and the connected components are
    calculated.\label{en:1}
    \item \textbf{Task API and 2-D Partitioning:} Data management is done
    without using the data-parallel API. The framework is used for task
    scheduling. Data are pre-partitioned in 2-D partitions and passed to a
    \texttt{map} function that calculates the edge list using \texttt{cdist},
    realized as a loop for MPI. The list is collected (gathered to rank 0) and
    the connected components are calculated.\label{en:2}
    \item \textbf{Parallel Connected Components:} Data are managed as in
    approach~\ref{en:2}. Each \texttt{map} task performs edge list and connected
    components computations. The reduce phase joins the calculated components
    into one, when there is at least one common node.\label{en:3}
    \item \textbf{Tree-based Nearest Neighbor and Parallel-Connected Components
    (Tree-Search):} This approach is different to approach~\ref{en:3} only for
    the way in which edge discovery is implemented in the \texttt{map} phase. A
    tree containing all atoms is created which is then used to query for
    adjacent atoms.\label{en:4}
\end{enumerate}

We use four physical systems with $131k$, $262k$, $524k$, and $4M$ atoms with
$896k$, $1.75M$, $3.52M$, and $44.6M$ edges in their graphs. We utilized up to
256 cores on Wrangler for our experiments. Data partitioning results into $1024$
partitions for each approach, thus $1024$ \texttt{map} tasks. Due to memory
limitations from using \texttt{cdist} -- uses double precision floating point --
Approach \ref{en:3} data partitioning of the $4M$ atom dataset resulted to $42k$
tasks for both Spark and MPI4py.

Figure \ref{fig:All4approachesNoRp} shows the runtimes for all datasets for
Spark, Dask and MPI4py. RP's performance is illustrated in
Figure~\ref{fig:rpLF}. We continue by analyzing the performance of each
architectural approach and used framework in detail.

\begin{figure}[t]
    \centering
    \begin{subfigure}{.85\textwidth}
        \centering
        \includegraphics[width=.95\linewidth]{figures/data_analytics_hpc/task_par/All4approachesWith4M_logscaleline.pdf}
    \end{subfigure}\\
    \begin{subfigure}{.85\textwidth}
        \centering
        \includegraphics[width=.95\linewidth]{figures/data_analytics_hpc/task_par/All4approachesWith4MSpeedup.pdf}
    \end{subfigure}
    \caption{Leaflet Finder Performance of Different Architectural Approaches
            for Spark \& Dask. Runtimes and Speedups for different system sizes
            over different number of cores for all approaches and frameworks.}
    \label{fig:All4approachesNoRp}
\end{figure}

%%%%%%%%%%%%%%% APPROACH 1 %%%%%%%%%%%%%%%%%%
\subsubsection*{Broadcast and 1-D Partitioning}

Approach 1 broadcasts data to nodes via the capabilities supported by Spark,
Dask and MPI. All nodes maintain a complete copy of the dataset and each
\texttt{map} task computes the pairwise distance on a partition of the dataset. 
%\mtnote{its
%partition of what? Note: 'its', 'it' and 'they' in English should not be used
%when it is not clear to what subject or complement they refers, especially at
%the beginning of a sentence. I have been correcting tens of those so far. In
%this sentence, it is not possible to understand to what 'its' referes: to the
%task, to the node, to the broadcast?}
%We use 1-D partitioning.
Figure~\ref{fig:WranglerLeafLetFinderApp1} shows the detailed results.

\begin{figure}[t]
    \centering
    \includegraphics[width=.75\textwidth]{figures/data_analytics_hpc/task_par/spark_dask_lf_approach1.pdf}
    \caption{Broadcast and 1-D Partitioned Leaflet Finder (Approach 1). Runtime
    for multiple system sizes on different number of cores for Spark, Dask and
    MPI4py.}
    \label{fig:WranglerLeafLetFinderApp1}
\end{figure}

The usage of broadcast capabilities has severe limitations for Spark and Dask.
MPI broadcast is a fraction of the overall execution time and significantly
smaller than Spark and Dask. MPI's broadcast times increase linearly as the
number of processes increases, while Spark's and Dask's remain relatively
constant for each dataset, due to more elaborate broadcast algorithms compared
to MPI. Broadcast times are between $3\%$ and $15\%$ of the edge discovery time
for Spark, from $40\%$ to $65\%$ for Dask, and from $<1\%$ to $10\%$ for MPI4py.
Thus, Spark offers a more efficient communication subsystem compared to Dask. In
addition, Dask broadcast partitions the dataset to a list where each element
represents a value from the initial dataset. This did not allow broadcasting the
$524k$ atom dataset. Nevertheless, the limited scalability of this approach due
to transmitting the entire dataset makes it usable only for small datasets.

Using broadcast shows the worst performance and scaling of all approaches for
Spark, Dask and MPI4py. This approach scales only up to $262k$ atoms for Dask,
and $524k$ atoms for Spark and MPI4py on Wrangler. Spark's performance is
comparable to MPI4py for the $262k$, and $524k$ datasets. Spark
also shows better performance
for the smallest core count in the $524k$ case. Dask is at least two times
slower than our MPI implementation.

%%%%%%%%%%%%%%% APPROACH 2 %%%%%%%%%%%%%%%%%%
\subsubsection*{Task-API and 2-D Partitioning}

Approach~\ref{en:2} tries to overcome the limitations of approach 1, especially
broadcasting and 1-D partitioning. A 2-D block partitioning is essential, as it
evenly distributes the compute and more efficiently utilizes the available
memory. 2-D partitioning is not well supported by Spark and Dask. Spark's RDDs
are optimized for data-parallel applications with 1-D partitioning. While Dask's
array supports 2-D block partitioning, it was not used for this implementation.
We return the adjacency list of the graph instead of an array to fully use the
capabilities of the abstraction. Thus, each task works on a 2-D pre-partitioned
part of the input data.

Figure~\ref{fig:All4approachesNoRp} shows the runtimes of approach~\ref{en:2}
for Spark, Dask, and MPI4py, and Figure~\ref{fig:rpLF} shows the runtime of
approach~\ref{en:2} for RP. As expected, this approach overcomes the
limitations of approach 1 and can easily scale to larger datasets (e.g., $524k$
atoms) while improving the overall runtime. Dask's execution time was smaller by
at least a factor of two compared to
approach 1. However, we were not able to scale this implementation to the 4M
dataset due to the memory requirements of \texttt{cdist}. For RP, we
observed significant task management overheads (see also
section~\ref{sec:framework_eval}). This is a limitation of RP with
respect to managing large numbers of tasks, particularly visible when running on
a single node with 32 cores. As more than 64 cores become available, 
RP's performance improves dramatically.

\begin{figure}[t]
    \centering
    \includegraphics[width=.75\textwidth]{figures/data_analytics_hpc/task_par/rpLF.pdf}
    \caption{RADICAL-Pilot Task API and 2-D Partitioned Leaflet Finder (Approach
    2). Runtime for multiple system sizes over different number of cores.}
    %Overheads dominate since execution times are similar despite the system size.}
    \label{fig:rpLF}
\end{figure}

Spark and Dask did not scale as well as MPI, which achieved linear speedups of
$\sim8$ when using $256$ cores. Spark and Dask achieved maximum speedups of
$4.5$ and $\sim5$ respectively. Despite this fact, both frameworks had similar
performance on $32$ cores for the $262k$ and $524k$ datasets.

%%%%%%%%%%%%%% Approach 3 %%%%%%%%%%%%%%%%%%%%%%%
\subsubsection*{Parallel Connected Components}

Communication between the edge discovery and connected components phases is
another important aspect. For the $524k$ atoms dataset, the output of the edge
discovery phase is $\approx$ $100\,\textup{MB}$. To reduce the amount of data
that need to be shuffled, we refined the algorithm to compute the graph
components on the partial dataset in the \texttt{map} phase. The partial
components are then merged in a \texttt{reduce} phase. This reduces the amount
of shuffle data by more than $50\%$ (e.g., to $12\textup{MB}$ for Spark and
$48\textup{MB}$ for Dask). Figure~\ref{fig:All4approachesNoRp} shows the
improvements in runtime, by $\sim20\%$ for Spark and Dask, but not MPI4py.
Further, we were able to run very large datasets, such as the 4M dataset, using
this architectural approach only with Spark and MPI4py. Dask was restarting its
worker processes because their memory utilization was reaching $95\%$.

Spark and Dask have comparable performance with MPI on 32 cores, which utilizes
a single node on Wrangler. However, while the MPI4py implementation scales
almost linearly for all datasets, Spark and Dask cannot, reaching a maximum of
$\sim5$ for the three smaller datasets. In addition, Spark is able to scale
almost linearly for the $4M$ atoms dataset, providing comparable performance to
MPI4py.

%%%%%%%%%%%%%% Approach 4 %%%%%%%%%%%%%%%%%%%%%%%
\subsubsection*{Tree-Search}

A bottleneck of approaches~\ref{en:1},~\ref{en:2} and~\ref{en:3} is the edge
discovery via the naive calculation of the distances between all pairs of atoms.
In approach~\ref{en:4}, we replace the pairwise distance function with a
tree-based, nearest neighbor search algorithm, in particular
BallTree~\cite{omohundro89five}. The algorithm:
\begin{inparaenum}
    \item constructs of a tree; and
    \item queries for neighboring atoms.
\end{inparaenum}

Using tree-search, the computational complexity can be reduced from $n^2$ to
$log$. We use a BallTree as offered by Scikit-Learn~\cite{scikit-nearest} for
our implementation.

Figure \ref{fig:All4approachesNoRp} illustrates the performance of the
implementation. For small datasets, i.e., $131k$ and $262k$ atoms,
approach~\ref{en:3} is faster than the tree-based approach, since the number of
points is too small. For the large datasets, the tree approach is faster. In
addition, the tree has a smaller memory footprint than \texttt{cdist}. This
allowed to scale to larger problems, e.g., a $4M$ atoms and $44.6M$ edges
dataset, without changing the total number of tasks.

Dask shows better scaling than Spark for $131k$, $262k$, and $524k$ atoms. This
is not true for $4M$ atoms, indicating that Dask's communication layer is not
able to scale as well as Spark's one. Spark shows similar performance to MPI4py
for the largest dataset due to minimal shuffle traffic. Thus, MPI's efficient
communication does not become relevant.

\section{Conceptual Framework for Selecting Task-Parallel Frameworks}
\label{sec:task_sel_model}

In this section, we provide a conceptual framework that allows application
developers to select a framework according to their compute and I/O
requirements. It is important to understand the properties of both the
application and task-parallel frameworks. Table~\ref{tab:framework} illustrates
the criteria of this conceptual model and ranks the three frameworks.

\begin{table}[t]
    \scriptsize
    \centering
    \begin{tabular}{@{}cccc@{}}
        \toprule
        &\textbf{RADICAL-Pilot}     &\textbf{Spark} &\textbf{Dask}\\
        \multicolumn{4}{l}{\textbf{Task Management}} \\
        \midrule
        Low Latency   &- &o &+\\
        Throughput    &- &+ &++\\
        MPI/HPC Tasks &+ &o &o\\
        Task API   &+ &o &++\\
        Large Number of Tasks   &-- &++ &++\\\hline
        \multicolumn{4}{l}{\textbf{Application Characteristics}}\\\midrule
        Python/native Code &++ &o &+\\
        Java               &o &++ &o\\
        Higher-Level Abstraction &- &++ &+\\
        Shuffle                  &- &++ &+\\
        Broadcast                &- &++ &+\\
        Caching                  &- &++ &o\\
        \bottomrule
    \end{tabular}
    \caption{Task-parallel framework selection decision methodology: Criteria
        and Ranking for Framework Selection. -~: Unsupported or low performance
        +~: Supported, ++~: Major Support, and o~:Minor
        support.\label{tab:framework}}
\end{table}

%  application perspective
\subsubsection*{Application Perspective}

We showed that we can implement applications for MD trajectory data analysis
using Spark, Dask and RP, as well as MPI4py. Implementation aspects,
such as computational complexity, and shuffled data size greatly influence
performance. For embarrassingly parallel applications with coarse grained tasks,
such as PSA, the choice of the framework does not significantly influence
performance. In addition, the performance difference against MPI4py was not
significant
(Figures~\ref{fig:HausdorffWrangler},~\ref{fig:comet_wrangler_haus}). Thus,
aspects such as programmability and integrate-ability, become more important
than performance alone.

For fine-grained data parallelism, a data-parallel framework, such as Spark and
Dask, clearly outperforms RP
(Figures~\ref{fig:All4approachesNoRp},~\ref{fig:rpLF}). If coupling is
introduced, i.e., task communication is required, using Spark becomes
advantageous (Approaches~\ref{en:3} \& \ref{en:4}). MPI4py outperformed
Dask and Spark, despite both frameworks scaling for the larger datasets.
Especially Spark was able to provide linear speedup for approach~\ref{en:3} of
Leaflet Finder (Figure~\ref{fig:All4approachesNoRp}).

Integrating with frameworks that provide higher level abstractions provides
scalable solutions for more complex algorithms. However, integrating Spark with
other tools needs to be carefully considered. The integration of Python tools,
e.g., MDAnalysis, often causes overheads due to the frequent need for
serialization and copying data between the Python and Java space.

Dask has the smallest learning curve of all three frameworks. As a result, it
allows for faster prototyping compared to RP and Spark.
RP's learning curve is more steep, but is more versatile than Dask
and Spark, by offering the lowest level abstraction. Spark had the slowest
learning curve but it required tuning to get the number of tasks correctly, as
well as argument passing to map and reduce functions.

\subsubsection*{Framework Perspective}

RP is well suited for HPC applications, e.g., ensembles (up to $50k$
tasks) of parallel MPI applications, as shown in
Ref.~\cite{merzky2018design,merzky2019using}. It has limited scalability when
supporting large numbers of short-running tasks, as often found in
data-intensive workloads. The file staging implementation of RP is
not suitable for supporting the data exchange patterns, i.e., shuffling, required
for these applications. However, concurrently executing MPI and Spark applications
on the same resource makes RP particularly suitable when different
programming models need to be combined.

% framework perspective
Dask provides a highly flexible, low-latency task management and excellent
support for parallelizing Python libraries. We established that Dask has higher
throughput than the other frameworks
(Figures~\ref{fig:dask_spark_rp_wrangler},~\ref{fig:RP_Dask_Spark_throughput}).
However, Spark provides better speedups for the largest datasets compared to
Dask (Figure ~\ref{fig:All4approachesNoRp}). Dask's broadcast (Leaflet Finder
with Approach~\ref{en:1}) and shuffle (Leaflet Finder with
Approaches~\ref{en:2}-~\ref{en:4}) performance is worse for larger problems
compared to Spark. Thus, Dask's communication layer shows some weaknesses that
are particularly visible during broadcast and shuffle. Spark needs to be taken
into consideration for shuffle-intensive applications. Its in-memory caching
mechanism is particularly suited for iterative algorithms that maintain a static
set of data in-memory and conduct multiple passes on that set.

\section{Conclusions}
\label{sec:tp_concl}

In this chapter, we investigated the use of different programming abstractions
and frameworks for implementing a range of algorithms for MD trajectory
analysis. We conducted an in-depth analysis of applications' characteristics and
assessed the architectures of RP, Spark and Dask. We provided a
conceptual framework that enables application developers to qualitatively
evaluate task parallel frameworks with respect to application requirements. Our
benchmarks enable developers to quantitatively assess framework performance as
well as the performance of different implementations. Our method can be used for
any application in which data are represented as time series of simulated
systems, e.g., weather forecast and earthquake simulation.

%Lesson Learned
While the task abstractions provided by all frameworks are well-suited for
implementing the considered use cases, the high-level MapReduce programming
model provided by Spark and Dask has several advantages. It is easier to use and
efficiently supports common data exchange patterns, e.g., shuffling between
\texttt{map} and \texttt{reduce} stages. In our benchmarks, Spark outperforms
Dask in communication-intensive tasks, such as broadcasts and shuffles. Dask
provides more versatile low and high level APIs and integrates better with
python frameworks. RP does not provide a MapReduce API, but is well
suited for coarse-grained task-level parallelism~\cite{merzky2018design,
merzky2019using}, and when HPC and analytics frameworks need to be integrated.
We also identified a limitation in Dask and Spark: while both frameworks provide
some support for linear algebra through a distributed array abstraction, it
proved inflexible for an efficient all-pairs pattern implementation. Dask and
Spark required workarounds and utilization of out-of-framework functions to read
and partition data (Table~\ref{tab:app_operators}). Although none of these
frameworks outperformed MPI, their scaling capabilities along with their
high-level APIs create a strong case on utilizing them for data analytics with
HPC applications.

Our results allow users to utilize a well-suited task-parallel framework based
on the characteristics of the data-intensive workflows of a computational
campaign. This in turn reduces the time to define the data-intensive workflows
of a computational campaign. Further, it allows the scalable execution of such
workflows without significant, if any, effort from users, reducing the time to
optimize a workflow. However, there is a set of data analysis workflows that do
not necessarily conform with the MapReduce abstraction, such as the PSA analysis
or earth science workflows that analyze imagery, as they have both data and
compute intensive characteristics. As a result, the selected framework should
efficiently and effectively perform data transfers and execution while offering
the necessary abstractions. In the next chapter we discuss architectural
equivalent designs of task-parallel frameworks to execute such data- and
compute-intensive workflows.
