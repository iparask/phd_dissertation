\label{ch:cmanager}

We discussed the requirements to support data-intensive workloads and workflows on high performance resources (HPC).
Executing, though, scientific computational campaigns asks for a software system that provides capabilities to plan and execute multiple workflows on multiple HPC resources.
There are software systems that are supporting computational campaigns, such as PanDA~\cite{maeno2008panda}, DIRAC~\cite{casajus2010dirac}, QCFractal~\cite{qcfractal}, glideinWMS~\cite{sfiligoi2008glidein}, and others. 
PanDA WMS~\cite{maeno2008panda}, is built to support the ATLAS experiment~\cite{atlas} at LHC, and is able to execute several thousand jobs concurrently processing million of tasks per week~\cite{de2015future}.
DIRAC~\cite{tsaregorodtsev2003dirac} is LHCb Monte Carlo production system at CERN.
QCFractal~\cite{qcfractal} is a campaign manager to execute upon large scale quantum chemistry data.
GlideInWMS~\cite{sfiligoi2008glidein} is a more general purpose campaign manager as it was designed to support different use cases.
These are domain specific campaign managers and make assumptions about the underlying software stack, and the type of workflows to be executed.
%These are domain specific campaign managers and make assumptions about the underlying software stack, and the type of workflows to be executed.
%Our proposed approach is domain agnostic, and makes no assumptions about workflows, or the software stack that is going to be used.

These campaign managers support a plethora of computing resources, including Grid resources, HPCs and Cloud.
PanDA~\cite{maeno2008panda}, glideinWMS~\cite{sfiligoi2008glidein}, and DIRAC~\cite{casajus2010dirac} mainly support grid resources.
PanDA has been extended to support HPC resources~\cite{de2015future, de2016accelerating}, and clouds~\cite{de2016accelerating}, while glideinWMS~\cite{sfiligoi2008glidein} supports clouds.
QCFractal~\cite{qcfractal} supports a set of heterogeneous resources including local campus clusters, HPC resources and clouds.
In addition, there are workflow management systems on HPCs, like Pegasus~\cite{deelman2015pegasus}, and Balsam~\cite{salim2019balsam}, that support the execution of multiple workflows on HPC resources, but not specifically campaigns.

In addition, campaign managers are making assumptions about the resources and the middleware they are utilizing, are monolithic software systems, despite their modular design, and tend to be domain specific.
For example, PanDA~\cite{maeno2008panda}, Pegasus~\cite{deelman2015pegasus}, and glideinWMS~\cite{sfiligoi2008glidein} are not easily extensible to use other capabilities or runtime systems.
QCFractal is built to be extensible and be able to interface with different workflow and workload management systems. Currently, it supports multiple execution engines.
This in turn forces domain scientists and user to either build custom tools that support their needs or fit their campaigns to a selected software system.

In response to these limitations, we designed and prototyped a new campaign manager (CM).
Our campaign manager supports use cases from different domains, such as molecular dynamics and earth sciences.
As a result, it is domain agnostic.
In addition, our campaign manager is design by following the building blocks approach~\cite{turilli2019middleware}.
In this way, it will be agnostic of the system used to manage the execution of the campaign workflows.
This, in turn, will allow our CM to make no assumptions about the resources on which the campaign workflows will be mapped and executed.

In this chapter, we discuss the campaign manager requirements as they are derived by the use cases that it supports.
We describe the building blocks approach, the design architecture and implementation of the campaign manager and how it aligns with the building blocks.
Finally, we characterize the overheads of the campaign manager.

\section{Campaign manager requirements}
The requirements for designing a campaign manager can be vast.
We use two real use cases to derive the requirements of the campaign manager prototype.
The first use case supports quantum chemistry campaigns and the second earth sciences.

The quantum chemistry use case (UC1) has O(1k) workflows to execute, and up to 1000 to be executing at any given point in time, to a number of different resources. 
The workflows execution time varies between half-core hours up to 100-core hours.
In addition, users have access to resources with several capabilities and not every workflow can be executed in any resources. 

During the campaign definition, the users may provide an initial workflow priority.
This priority may change during runtime, as users want specific workflows to execute before others.
Furthermore, users may want to early bind workflows to resources, as some resources may already support the required software for a workflow, or have the necessary data there.

The set of resources the user has access to and their availability may change during the lifetime of a campaign.
User may get access to new resources as the campaign is executing, and would like to utilize them for the campaign.
In addition, a resource may become permanently unavailable as the user may lose access while the campaign executes.
The campaign may change during runtime as users add or remove workflows.

The earth science use case (UC2) requires the execution of multiple workflows to analyze images from different calendar years.
Workflows are a set of pipelines over an acquired dataset.
Workflow execution time varies from hours to a couple of days.
Workflows are added to the campaign as data are becoming available.
In addition, due to the data volume this use case requires a shared filesystem between the used resources.

A summary of the functional requirements is shown in Table~\ref{tab:fun_reqs}.

\begin{table}[t]
    \centering
    \scriptsize
    \begin{tabular}{@{}p{1.5cm}|p{2.8cm}p{1.5cm}p{6cm}@{}}
        \toprule
        \textbf{REQ ID} &\textbf{Requirement Algorithm} &\textbf{Use Case} & \textbf{Description} \\
        \midrule
         1 & 
         Support campaign with O(1k) workflows & 
         UC1 & 
         The campaign manager should be able to support campaigns with order of thousand workflows.
         Planning, execution and adaptation should be able to execute with such a campaign.\\
         2 & 
         The CM should support at least two different planning algorithms. & 
         G & 
         Users/developers should be able to easily extend the planning capabilities with algorithms.\\
         3 & 
         Support from 1 up to 100 resources & 
         UC1-UC2 & 
         The CM should be able to execute workflows on multiple resources concurrently.
         These resources can either be actual or emulated resources.\\
         4 & 
         Plan should be derived for heterogeneous and homogeneous static resources in 5 minutes & 
         G & 
         The plan should be derived as soon as the user provides a campaign description. 
         Plan should be derived in less than 5 minutes.\\
         5 & 
         Plan should be derived and adapted for heterogeneous/homogeneous dynamic resources in 5 minutes & 
         UC1 & 
         The plan should be derived as soon as the user provides a campaign description.
         Plan should be derived in less than 5 minutes.
         In case the plan needs to be adapted, it should be adapted in less than 5 minutes.\\
         6 & 
         Interface with different WMFs & 
         G & 
         The campaign manager should be able to interface with different WMFs based on the specifics of the campaign. \\
         7 & 
         Early bind workflows to resources &
         UC1 &
         The user may need to bind workflows to specific resources before executing the campaign.\\
         8 & 
         Campaign objective is configurable & 
         UC1 & 
         While the campaign is executing, the objective may be adjusted based on user preferences.\\
         9 &
         Update the campaign during runtime &
         UC1-UC2 &
         The user may want to update the campaign while it is executing to add/remove workflows.\\
         10 &
         Resources may be added or removed during runtime & 
         UC1 & 
         The user may want to add/remove resources, she has gained/lost access to.\\
        \bottomrule
    \end{tabular}
    \caption{Campaign manager functional requirements.\label{tab:fun_reqs}}
\end{table}




\section{Campaign Manager Design}
\subsection{Building Blocks Design Approach}
\subsection{Campaign Manager Components}
\section{Implementation and Overhead Evaluation}
