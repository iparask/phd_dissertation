\abstract{Many scientific application workflows are becoming larger and need to 
be executed multiple times with different parameters, generating a scientific 
computational campaign. Domains such as molecular dynamics, biological, and 
environmental sciences, define workflows and pipelines that are executed 
$O(1k)$ to $O(10k)$ times. In addition, based on results obtained during 
runtime the application computational requirements may change. Managing 
computational resources to reduce time to completion, maximize utilization, and 
maximize scientific insight, is becoming necessary. In this proposal, we 
present research done to understand:
    \begin{inparaenum}[(i)]
        \item scientific application time to completion characterization,
        \item scalability behavior of scientific workflows based on programming 
        models.
       \end{inparaenum}
We summarize preliminary results from our research for modeling scientific 
workflows in terms of time to completion, and propose an autonomic workload 
manager to manage workflow execution on High Performance resources. We include 
a work plan to investigate, formulate and implement the proposed workload 
manager and evaluate foreseen challenges and risks}