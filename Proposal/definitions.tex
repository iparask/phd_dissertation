\label{definitions}
Terms such as ``execution plan'', ``workflow'', ``computational campaign'', and others are usually overloaded in literature and defined differently across domains. We define the terminology used in this proposal:
\begin{itemize}
    \item \textbf{Computational Campaign} is a set of heterogeneous workflows with or without dependencies amongst them, that need to be executed to achieve a computational objective.
    \item \textbf{Workflow} represents the computational instance implementing an, or part of, application with specific parameter values, number of tasks, task dependencies, and other computational aspects.
    \item \textbf{Makespan} of a campaign is the time needed to execute all the workflows of the campaign, or alternatively, the maximum execution time among all paths throughout the campaign~\cite{chirkin2017execution}
    \item \textbf{Computational Objective} is a set of values selected by the user for a set of metrics which can be represented as an objective function.
    \item \textbf{Requirements} describe the minimum amount and type of resources needed to execute  each workflow of the campaign.
    \item \textbf{Constraints} are the conditions that bound the execution, including but not limited to, resource availability, capacity or costs.
    \item \textbf{Execution plan} is a sequence of actions that solves the objective function as, for example, selecting, acquiring and configuring resources, and establishing the execution order of workflows.
\end{itemize}
