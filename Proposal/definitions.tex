Terms such as ``workflow'', ``workload'', ``computational campaign'', and 
``autonomic computing'' are usually overloaded in literature and defined 
differently across domains. We define the terminology used in this proposal:
\begin{itemize}
    \item \textbf{Autonomic Computing}: Based on the definition by 
    IBM~\cite{ibm2005autonomic}, it is a computing environment with the ability 
    to manage itself and dynamically adapt to change in accordance with 
    business policies and objectives.
    \item \textbf{Application} specifies a single or multiple tasks, and the 
    interaction and relationship (or lack thereof) among multiple tasks. The 
    task(s), along with their interactions and relationships represent an 
    algorithmic solution to a science problem.
    \item \textbf{Workflow} represents the computational instance implementing 
    an, or part of, application with specific parameter values, number of tasks, 
    task interdependencies, and other computational aspects.
    \item \textbf{Workload} refers to a set of tasks whose dependencies have 
    been satisfied and are ready to be concurrently executed. As such, 
    workloads can be subsets of the tasks of a workflow. The “load” in workload 
    is a reference to a measure of the computational requirements of each 
    individual task. This is relevant when scheduling a workload heterogeneous 
    tasks.
    \item \textbf{Computational Campaign} refers to a set of distinct workflows 
    executed over a heterogeneous set resources multiple times.
    \item \textbf{Task} is a stand-alone process that has well defined input, 
    output, termination criteria, and resources requirements.
\end{itemize}