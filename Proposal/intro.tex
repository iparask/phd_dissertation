Scientific applications are benefiting from defining ensembles of computational 
tasks whose collective impact provide insight to the studied problem. These 
tasks are organized in pipelines with well defined temporal and data 
dependencies, and create complex workflows with hundreds or thousands ensemble 
members~\cite{malawski2015algorithms,rietmann2012forward,dakka2018high}. These 
workflows are either well known prior to execution, and as a result static, or
change by adapting tasks to ensemble member and/or ensemble members of the 
workflow, thus dynamic. Static~\cite{paraskevakos2019workflow} and 
dynamic~\cite{dakka2018high} workflows are used and applied in many scientific
domains, e.g. ecological and biomolecular sciences. 

There is a set of workflows, either static or dynamic, which have a sufficient 
number of tasks and dependencies that cannot be executed in a single machine as 
a unit. The execution of such workflows is partitioned in multiple, independent 
or dependent, executions. In addition, these executions may need to be repeated 
multiple times based on intermediate result analysis. Biomolecular science 
workflows, for example, analyze data locally in ensemble members or globally and 
adapt by adding tasks, adding/removing ensemble members, or execute the whole 
workflow with different initial conditions. This departs from the approach where 
a user submits a workflow and waits for the final results. The user monitors 
the execution and submits partitions of the workflow for execution, while 
coordinating data analysis and workflow adaptation. This way of executing a 
workflow is called computational campaign execution.

The execution of a computational scientific campaign requires from users to 
derive and realize an execution plan. Several factors influence this plan, 
including resource availability, and capabilities, as well as workflow partition 
requirements. Resource availability refers to the resources the user has 
currently access and can use for execution. Resource capabilities include: 
\begin{inparaenum}[1)]
\item offered computing capacity in terms of number of CPU cores, and possible 
      accelerators, such as GPUs,
\item offered main memory size, and
\item offered filesystem in terms of size, and whether it is shared or not.
\end{inparaenum} The user requires to evaluate such factors and partition the 
workflow such that each partition's requirements are satisfied by the 
capabilities of available resources. In addition, users may update the plan 
during execution based on changes on resource availability, and capabilities.

High Performance Computing (HPC) resources have a broad spectrum of usage 
policies affecting resource availability, and execution on such resources. These 
policies determine:
\begin{inparaenum}[1)]
\item the maximum number of compute nodes and walltime a user can request for a 
single job, i.e. execution,
\item the maximum number of concurrently submitted jobs (either executing or 
waiting in the scheduler's queue), and
\item one or more usage charging models 
\end{inparaenum}. Execution on such resources requires an allocation which is, 
for example, measured in core-hours, defining the computing capacity available 
to the user. The user has, based on resource policies, available capacity, and 
workflow performance metrics to evaluate how such capacity satisfies the 
campaign execution over the allocation period. These are complex and time 
consuming evaluations that depend on many variables, including resource usage 
policies, amount of resources consumed by the campaign’s workflow, and 
performance metrics for the workflow execution.
 
Campaign workflow partitioning requires a model that can provide estimations 
of the total execution time of each partition, based on single task execution 
time, achievable degree of task execution concurrency, and execution 
coordination overheads. A model can be used to estimate the execution time of 
each partition, given single task execution distributions, and desirable 
degree of concurrency. The overheads of the execution depend on the middleware 
used to execute the campaign. Analyzing and modeling of the overheads of 
different architectural approaches can provide an estimation, and the type of 
approach that is preferable for each partition. Combining the two models can 
provide an estimation of the time to completion of each workflow partition. 
 
Determining the times workflow partitions can be executed, based on an 
allocation, requires a cost estimation function. Provided a time to completion 
model, and a usage charging policy, the user can derive such a cost function 
for the execution of the workflow partitions. Based on this cost function and a 
set of resource, the user can evaluate how the allocation will be utilized for 
every selected resource. Using the utilization evaluation, the user can then 
optimize a selected metric, such as, maximizing number of executions, 
maximizing concurrency, minimizing execution cost, and defining an execution 
plan for their computational campaign.
 
Scientific workflows are mainly executed by utilizing dedicated workflow 
management frameworks (WMF), such as RADICAL-Ensemble 
Toolkit~\cite{balasubramanian2018harnessing}, Pegasus~\cite{deelman2015pegasus}, 
Kepler~\cite{ludascher2006concurrency}, and others. These frameworks offer 
runtime capabilities, such as task execution, data dependency resolution, and 
workflow definition and monitoring. Given a set of resources and a walltime, 
WMF try to maximize resource utilization and minimize time to completion. WMF 
assume that the user selects sufficient resources and walltime to execute the 
workflow. Some workflow management frameworks, such as 
Dask~\cite{rocklin2015dask} and Airflow~\cite{airflow}, provide capabilities to 
elastically adapt resources, by scaling up or down, based on the current state 
of execution. They do not offer capabilities that allow execution planning and 
workflow partitioning. Deriving, validating and adapting an execution plan 
requires from the user expertise outside of their scientific domain. Providing 
a workflow management system with capabilities to plan, monitor, and adapt the 
execution becomes therefore desirable. Such a system can utilize models of 
workflow time to completion, alongside workflow execution capabilities, to 
automate campaign workflow partitioning and execution.

In this proposal, I propose an autonomic workflow manager for executing 
scientific computational campaigns on high performance computing resources. 
Autonomic systems, as defined by IBM~\cite{ibm2005autonomic}, provide 
capabilities of self-monitoring, self-configuring, and self-regulating. An 
autonomic workflow manager will be able to define an initial execution plan 
(self-configuring), monitor its effectiveness (self-monitoring), and adjust 
during runtime (self-regulating). During the execution of this proposal, we 
will:
\begin{inparaenum}[1)]
\item evaluate and derive time to completion models for a set of workflows 
provided by ecological and biomolecular sciences use cases,
\item offering execution planning capabilities, and 
\item exploring the requirements to support execution plan adaptivity
\end{inparaenum}.

The proposal is organized as follows. Section 2 provides definitions for the 
used terminology. Section 3 introduces research done providing a unified 
runtime system that supports data analytics frameworks on HPC resources, and 
creating a conceptual model for the performance of different workflow management 
systems for a set of data analysis workflows. Section 4 presents initial research 
done towards modeling the time to execution of a scientific workflow. Section 5 
presents the research required to design and implement the proposed workflow 
manager. It includes the methodology that will be used to model scientific 
workflows, and architectural details of the manager. Section 6 concludes the 
proposal by describing the proposal's timeline, significance and impact of the 
work, and the challenges of this work.