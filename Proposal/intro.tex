Scientific applications are benefiting from executing multiple workflows, with or without dependencies amongst them, to achieve scientific insight.
Biomolecular sciences, for example, execute workflows simulating physical systems with different initial conditions to enable better drug discovery~\cite{dakka2018concurrent}.
Ecological sciences that use very high resolution (VHR) satellite imagery, require the analysis of TB of data from different calendar years to create time series of ecological changes.
This departs from the approach where the user submits a single workflow and wait for the final results.
Instead the user monitors the execution of several workflows, selects resources, and submits workflows for execution, while coordinating data analysis and workflow adaptation.
This way of execution is called executing a computational campaign.

Computational campaigns enact an execution plan to allow users to achieve a computational objective under given requirements and constraints.
A computational objective is a set of values selected by the user for a set of metrics, e.g., time to completion and throughput, which can be represented as an objective function.
Requirements describe the minimum amount and type of resources needed to execute each workflow of the campaign, while constraints are the conditions that bound the execution, including but not limited to, resource availability, capacity or costs.
A plan describes a sequence of actions that solves the objective function as, for example, selecting, acquiring and configuring resources, and establishing the execution order of workflows on resources.


Planning and enacting the execution of a campaign poses four main challenges: 
\begin{inparaenum}[(i)]
\item evaluating the makespan of a campaign on heterogeneous and dynamic resources;
\item the conditions under which an execution plan performs better compared to a random resource selection;
\item determining a campaign execution plan on available resources that satisfies the given objective function, requirements and constraints of a campaign, while accounting for resource dynamism; and
\item adapting the plan in case of deviation from the objective achievement.
\end{inparaenum}

In this proposal, we propose a campaign manager for executing scientific computational campaigns on high performance computing resources.
This campaign manager, given a campaign, an objective and a set of constraints, will derive and enact upon an execution plan. 
In case, the plan deviates from achieving the objective of the campaign the campaign manager will adjust it.

During the execution of this proposal, we will:
\begin{inparaenum}[(1)]
\item select an algorithm to calculate the makespan of a campaign and determine an execution plan on possibly heterogeneous and dynamic resources;
\item a software system that implements the makespan algorithm, and executes a campaign on resources; and 
\item a method to evaluate the performance of our approach compared to a random plan. 
\end{inparaenum}

The proposal is organized as follows.
Section~\ref{definitions} provides definitions for the used terminology.
Section~\ref{current_research} provides an overview of the research done to support scientific data-intensive workflows on HPC resources, as well as the state of the art on executing scientific campaigns on resources.
Section~\ref{research} presents the research required to design and implement the proposed campaign manager.
Section~\ref{timeline} concludes the proposal by describing the proposal's timeline, significance and impact of the work, and the challenges of this work.