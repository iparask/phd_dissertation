\documentclass{article}

%% Language and font encodings
\usepackage[english]{babel}
\usepackage[utf8x]{inputenc}
\usepackage[T1]{fontenc}
\usepackage{float}
\usepackage{listings}
\usepackage{nicefrac}
%% Sets page size and margins
\usepackage[letterpaper,top=3cm,bottom=2cm,left=3cm,right=3cm,marginparwidth=1.75cm]{geometry}

%% Useful packages
\usepackage{amsmath}
\usepackage{amsthm}
\usepackage{graphicx}
\usepackage{color}
\usepackage{xcolor}
\usepackage{cleveref}
\usepackage{booktabs}
\usepackage{multirow}
\usepackage{paralist}
%\definecolor{codegreen}{rgb}{0,0.6,0}
%\definecolor{codegray}{rgb}{0.5,0.5,0.5}
%\definecolor{codepurple}{rgb}{0.58,0,0.82}
%\definecolor{backcolour}{rgb}{0.95,0.95,0.92}

\newif\ifdraft
\drafttrue
\ifdraft
\definecolor{ocolor}{rgb}{1,0,0.4}
\newcommand{\jwave}[1]{ {\reduwave{#1}}}
\newcommand{\jhanote}[1]{ {\textcolor{red} { ***shantenu: #1 }}}
\newcommand{\mtnote}[1]{ {\textcolor{cyan} { ***matteo: #1 }}}
\definecolor{orange}{rgb}{1,.5,0}
\definecolor{dandelion}{cmyk}{0,0.29,0.84,0}
\newcommand{\gpnote}[1]{{\textcolor{green} {***giannis: #1}}}
\newcommand{\note}[1]{ {\textcolor{magenta} { ***Note: #1 }}}
\else
\newcommand{\jwave}[1]{}
\newcommand{\jhanote}[1]{}
\newcommand{\mtnote}[1]{}
\newcommand{\gpnote}[1]{}
\newcommand{\note}[1]{}
\fi

\theoremstyle{definition}
\newtheorem{defn}{Definition}[section]

\lstdefinestyle{mystyle}{
    backgroundcolor=\color{backcolour},   
    commentstyle=\color{codegreen},
    keywordstyle=\color{magenta},
    numberstyle=\tiny\color{codegray},
    stringstyle=\color{codepurple},
    basicstyle=\footnotesize,
    breakatwhitespace=false,         
    breaklines=true,                 
    captionpos=b,                    
    keepspaces=true,                 
    numbers=left,                    
    numbersep=5pt,                  
    showspaces=false,                
    showstringspaces=false,
    showtabs=false,                  
    tabsize=2
}
 
\lstset{style=mystyle}
\title{Campaign manager for executing scientific campaigns on High Performance Computing resources}\author{Ioannis Paraskevakos}

\begin{document}
\maketitle

Progress in many scientific problems requires executing an increasing number of computational workflows to achieve scientific insight. We are motivated by workflows of ensembles of up to $O(10^5)$ computational tasks from ecological and biomolecular domains, organized in up to $O(10^3)$ pipelines~\cite{rietmann2012forward, dakka2018high, paraskevakos2019workflow} with heterogeneous resource requirements. Based on our use cases, a set of workflows is required to be executed, where each workflow can process and/or produce 100TB of data and execute tasks running for up to 24 hours. This set of workflows, with or without dependencies amongst them, constitutes a computational campaign. Effective and efficient execution of computational campaigns requires resource management and coordination at runtime.

A computational campaign is a set of heterogeneous workflows, with or without dependencies amongst them, that need to be executed. Workflows are heterogeneous when they have to either perform different types of work and/or have different resource requirements. We identify three types of dependencies between workflows in a campaign:
\begin{inparaenum}[1)]
\item data dependencies, 
\item temporal dependencies, and 
\item resource sharing dependencies
\end{inparaenum}. 
Temporal dependencies refer either to a temporal order between workflows or to a moment in time where the workflow can be executed. Resource sharing dependencies between workflows refer to workflows executing in the same resource concurrently given sufficient capacity or interchangeably. For a set of workflows to be categorized as a campaign it should have at least ten workflows.

Workflows are mainly executed by utilizing dedicated workflow management frameworks~\cite{balasubramanian2018harnessing,deelman2015pegasus,ludascher2006scientific,rocklin2015dask,airflow}. Currently, workflow management frameworks offer a vast array of capabilities but, tend to lack campaign planning and execution capabilities. Campaign planning and execution requires scheduling workflows to resources, resolving dependencies between workflows, as well as managing the execution of the campaign. As a result, when executing a computational campaign, users have to derive, execute and often adapt campaign execution plans, manually solving scheduling problems, satisfying dependencies, and managing workflow executions. Thus, a campaign manager to automatically derive, execute and adapt an execution plan is desirable.

Resources, workflows and campaigns can either be static or dynamic. A resource is considered static when its performance is constant over time, or dynamic when it changes. A workflow can be known \textit{a-priori}, i.e. all tasks and dependencies described, and thus static~\cite{paraskevakos2019workflow}, or change during runtime by adding tasks and dependencies, thus dynamic~\cite{dakka2018high}. Accordingly, when the set of workflows, and their dependencies, that comprise the campaign is known \textit{a-priori}, the campaign is considered static, and when the set of workflows change, by adding workflows, it is considered dynamic.

Computational campaigns enact an execution plan to allow users to achieve an objective under given requirements and constraints. A computational objective is a set of values selected by the user for a set of metrics, such as time to completion and throughput. This objective can then be represented as an objective function. Requirements describe the minimum amount and type of resources needed to execute each workflow of the campaign, while constraints are the conditions that bound the execution of the campaign, including but not limited to, resource availability, capacity, and costs. A plan describes a sequence of actions such as selecting, acquiring and configuring resources, that solve the objective function, as well as dependency resolution.

Planning the execution and executing a computational campaign poses three main challenges: 
\begin{inparaenum}[(i)]
\item evaluate the time needed to execute all the workflows, i.e. makespan, of a campaign on heterogeneous and dynamic resources,
\item determining a plan between workflows of a campaign and available resources that satisfies an objective function while being subject to the requirements and constraints of a campaign, and
\item adapting the plan in case of deviating from the objective achievement.
\end{inparaenum}

In accordance with those challenges and within the requirements of our target use cases, this work will offer three main contributions: 
\begin{inparaenum}[(1)]
\item an algorithm to calculate the makespan of a campaign on heterogeneous resources;
\item a software system that implements the makespan algorithm and executes a campaign on resources; and 
\item a method to evaluate the performance of our approach compared to a random plan. 
\end{inparaenum}
Compared to existing solutions, our research will allow users to execute campaigns that contain $O(10)$ workflows with $O(1000)$ tasks. We will implement these contributions into a software prototype designed to execute ecological and biomolecular use cases on high performance computing (HPC) resources, using an existing workflow execution framework as implemented by RADICAL-EnsembleToolkit (EnTK)~\cite{balasubramanian2018harnessing}.

The makespan of a campaign is the time needed to execute all the workflows of the campaign, or alternatively, the maximum execution time among all paths throughout the campaign~\cite{chirkin2017execution}. Several methods have been proposed~\cite{lu2019review} to calculate the makespan of a workflow, including queuing network~\cite{yao2019throughput,bao2019performance}, domain specific languages~\cite{carothers2017durango,maheshwari2016workflow}, and mathematical programming~\cite{liu2017mathematical}. Heterogeneous Earliest Finish Time (HEFT)~\cite{topcuoglu2002performance} is an offline scheduling algorithm which calculates the makespan of a workflow on heterogeneous resources. HEFT uses a matrix to represent execution time of tasks on resources. HEFT assigns tasks to the resource that minimizes the finish time of the task. HEFT has complexity proportional to the number of dependencies between tasks and the number of processors offered. Furthermore, there has been some initial research to extend HEFT to resources that provide CPU and GPUs~\cite{shetti2013optimization}. We propose to extend HEFT algorithm to calculate the makespan of a computational campaign.

Calculating the makespan of a computational campaign depends on workflow characteristics, workflow dependencies, campaign dynamicity, and resource availability and dynamicity. Task execution time depends on parallelism, coordination between tasks, task characteristics~\cite{khoshlessan2017parallel}, the framework used to support task execution~\cite{paraskevakos2018task}, and resource dynamicity~\cite{paraskevakos2019workflow}. We propose to utilize approximations of task execution time that we will experimentally derive.

Based on workflow tasks’ characteristics, resources may need to be configured with different execution engines. Compute intensive tasks on HPC resources are mainly executed via OpenMP/MPI benefitting from parallelism at scale, while data intensive tasks via data parallel frameworks such as Hadoop, Spark~\cite{zaharia2010spark}, or Dask~\cite{rocklin2015dask}. Traditionally, HPC resources are designed to optimize the execution of MPI tasks, leaving data-parallel frameworks largely unsupported. In Ref.~\cite{luckow2016hadoop}, we show how a pilot-based middleware~\cite{merzky2019using} can support the efficient and scalable execution of data-parallel applications on HPC resources. Such capabilities are necessary to support and plan the execution of a data-intensive campaign.


\begin{figure}[t]
	\centering
	\includegraphics[width=.85\textwidth]{Proposal/extended_abstract/CEM_RefArch.pdf}
	\caption{Reference Architecture of a Campaign Execution Manager. Basic components of Campaign Execution Framework (CEF): 1) Campaign Execution Manager, subcomponents: i) Makespan Calculation, and Campaign, 2) Executor. CEM communicates decisions to RADICAL-EnTK. CEF communicates with HPCs to execute parts of the campaign.}\label{refarch}
\end{figure}

We propose to define a campaign manager (CM) which, given a campaign, an objective, and a set of constraints, can derive an execution plan by utilizing the proposed makespan methods. Figure 1 shows a reference architecture of the proposed framework. Fig.~\ref{refarch} shows a reference architecture of the framework. We propose to define a campaign manager (CM). CM has two sub-components: 
\begin{inparaenum}[(1)]
\item Makespan Calculation which implements the proposed makespan algorithm, and 
\item an Executor which executes the plan. Workflow execution is done through RADICAL-EnTK to HPC resources
\end{inparaenum}.
Based on workflows metrics, such as tasks execution time, overheads calculation and time to completion, provided by EnTK,  and as a consequence the respective workflows and campaign metrics, CEM will adapt the execution plan if necessary by updating the workflows to resource mapping decisions.

To understand the performance of the proposed approach, we will compare it with a randomly decided execution plan. We propose to model a random decision as the mapping of a campaign’s workflows to resources based on a uniform distribution between resources. The random plan provides a baseline of performance. Based on this comparison, we will be able to validate whether our approach offered better performance, i.e. smaller makespan after execution, as well as to better understand the problem’s requirements for further research and development.

Given the size of the problem space and that the proposed work will be performed in no more than one year, we propose to focus on the three main contributions described above. In more detail, we will focus on: 
\begin{inparaenum}[(1)]
\item evaluating and deriving the makespan for a campaign as a set of independent static O(10) workflows with heterogeneous resource requirements provided by ecological and biomolecular sciences use cases on static resources; 
\item offering execution planning capabilities to minimize the makespan of a campaign; and 
\item validating our planning capability by executing the workflows of our use cases and measuring the accuracy of the estimated campaign runtime and planned execution
\end{inparaenum}. In addition, we will explore the requirements to support execution plan adaptivity.

To this end, we propose to achieve the following objectives with an estimation of the time needed:
\begin{enumerate}
    \item Derive the makespan for a campaign from the ecological sciences. Durations 3 months.
    \item Develop a campaign manager, which will derive and execute a plan based on the results of objective 1. Duration 4 months, overlap with the last month of objective 1.
    \item Integration with scientific workflows and experimentally validate the plan by measuring the average deviation of the manager from the optimal objective. Duration 4 months. The first two months overlap with objective 2.
\end{enumerate}
\bibliographystyle{unsrt}
\bibliography{ext}
\end{document}