\documentclass{article}

%% Language and font encodings
\usepackage[english]{babel}
\usepackage[utf8x]{inputenc}
\usepackage[T1]{fontenc}
\usepackage{float}
\usepackage{listings}
\usepackage{nicefrac}
%% Sets page size and margins
\usepackage[letterpaper,top=3cm,bottom=2cm,left=3cm,right=3cm,marginparwidth=1.75cm]{geometry}

%% Useful packages
\usepackage{amsmath}
\usepackage{amsthm}
\usepackage{graphicx}
\usepackage{color}
\usepackage{xcolor}
\usepackage{cleveref}
\usepackage{booktabs}
\usepackage{multirow}
\usepackage{paralist}
%\definecolor{codegreen}{rgb}{0,0.6,0}
%\definecolor{codegray}{rgb}{0.5,0.5,0.5}
%\definecolor{codepurple}{rgb}{0.58,0,0.82}
%\definecolor{backcolour}{rgb}{0.95,0.95,0.92}

\newif\ifdraft
\drafttrue
\ifdraft
\definecolor{ocolor}{rgb}{1,0,0.4}
\newcommand{\jwave}[1]{ {\reduwave{#1}}}
\newcommand{\jhanote}[1]{ {\textcolor{red} { ***shantenu: #1 }}}
\newcommand{\mtnote}[1]{ {\textcolor{cyan} { ***matteo: #1 }}}
\definecolor{orange}{rgb}{1,.5,0}
\definecolor{dandelion}{cmyk}{0,0.29,0.84,0}
\newcommand{\gpnote}[1]{{\textcolor{green} {***giannis: #1}}}
\newcommand{\note}[1]{ {\textcolor{magenta} { ***Note: #1 }}}
\else
\newcommand{\jwave}[1]{}
\newcommand{\jhanote}[1]{}
\newcommand{\mtnote}[1]{}
\newcommand{\gpnote}[1]{}
\newcommand{\note}[1]{}
\fi

\theoremstyle{definition}
\newtheorem{defn}{Definition}[section]

\lstdefinestyle{mystyle}{
    backgroundcolor=\color{backcolour},   
    commentstyle=\color{codegreen},
    keywordstyle=\color{magenta},
    numberstyle=\tiny\color{codegray},
    stringstyle=\color{codepurple},
    basicstyle=\footnotesize,
    breakatwhitespace=false,         
    breaklines=true,                 
    captionpos=b,                    
    keepspaces=true,                 
    numbers=left,                    
    numbersep=5pt,                  
    showspaces=false,                
    showstringspaces=false,
    showtabs=false,                  
    tabsize=2
}
 
\lstset{style=mystyle}
\title{Campaign manager for executing scientific campaigns on High Performance 
Computing resources}
\subtitle{Extended Abstract}
\author{Ioannis Paraskevakos}
\vspace{-10ex}
\date{}
\begin{document}
\maketitle
\vspace{-7ex}
Progress in many scientific problems requires executing an increasing number of 
computational workflows to achieve scientific insight. We are motivated by 
workflows of ensembles of up to $O(10^5)$ computational tasks from ecological 
and biomolecular domains, organized in up to $O(10^3)$ 
pipelines~\cite{rietmann2012forward, dakka2018high, paraskevakos2019workflow} 
with heterogeneous resource requirements. Our use cases require to execute a set of 
workflows, where each workflow can process and/or 
produce up to 100TB of data and execute tasks running for up to 24 hours. This set 
of workflows can be characterized as a computational campaign. Effective and 
efficient execution of computational campaigns requires resource management 
and coordination at runtime.

A computational campaign is a set of heterogeneous workflows, with or without 
dependencies amongst them, that need to be executed. Workflows are heterogeneous 
when they perform different types of work and/or have different resource 
requirements. We identify three types of dependencies between workflows in a 
campaign:
\begin{inparaenum}[1)]
\item data, 
\item temporal, and 
\item resource sharing.
\end{inparaenum} 
Temporal dependencies refer either to a temporal order between workflows or to 
a moment in time where workflows can be executed. Resource sharing 
dependencies between workflows refer to workflows using the same resources 
either concurrently given sufficient capacity or interchangeably. 
% For a set of workflows to be categorized as a campaign it should have at least ten workflows.

Workflows are mainly executed by utilizing dedicated workflow management 
frameworks (WMF)~\cite{balasubramanian2018harnessing,deelman2015pegasus,ludascher2006scientific,rocklin2015dask,airflow}.
Currently, WMFs offer a vast array of capabilities 
but tend to lack campaign planning and execution capabilities. Campaign planning 
and execution requires scheduling workflows to resources, resolving dependencies 
between workflows, and managing the execution of the campaign. As a result, 
when executing a computational campaign, users have to derive, execute and often 
adapt campaign execution plans, manually solving scheduling problems, satisfying 
dependencies, and managing workflow executions. Thus, a campaign manager to 
automatically derive, execute and adapt an execution plan is desirable.

Resources, workflows and campaigns can either be static or dynamic. A resource 
is considered static when its performance is constant over time, or dynamic when 
it changes. A workflow can be known \textit{a-priori}, i.e. all tasks and 
dependencies are available before execution, and thus static~\cite{paraskevakos2019workflow}, or 
change during runtime by adding tasks and dependencies, thus dynamic~\cite{dakka2018high}. 
Accordingly, a campaign is static when the set of workflows and their dependencies is known \textit{a-priori}; dynamic 
when the set of workflows changes by adding workflows during the execution of the campaign.

Computational campaigns enact an execution plan to allow users to achieve a computational 
objective under given requirements and constraints. A computational objective is 
a set of values selected by the user for a set of metrics, e.g., time to 
completion and throughput, which can be represented as an objective 
function. Requirements describe the minimum amount and type of resources needed 
to execute each workflow of the campaign, while constraints are the conditions 
that bound the execution, including but not limited to, resource 
availability, capacity or costs. A plan describes a sequence of actions that solves the objective function as, for example, selecting, acquiring and configuring resources, and establishing the execution order of workflows.

Planning and enacting the execution of a campaign poses four main 
challenges: 
\begin{inparaenum}[(i)]
\item evaluating the time needed to execute the campaign, i.e., the makespan, on 
possibly heterogeneous and dynamic resources;
\item  understanding the conditions under which an execution plan performs 
better compared to a random resource selection;
\item determining a campaign execution plan on available resources that 
satisfies the given objective function, requirements 
and constraints of a campaign, while accounting for resource dynamism and performance 
variability; and
\item adapting the plan in case of deviation from the objective achievement.
\end{inparaenum}

In accordance with those challenges and within the requirements of our target 
use cases, this work will offer three main contributions: 
\begin{inparaenum}[(1)]
\item an algorithm to 
calculate the makespan of a campaign and determine an execution plan on possibly 
heterogeneous and dynamic resources;
\item a software system that implements the makespan algorithm, and executes a 
campaign on resources; and 
\item a method to evaluate the performance of our approach compared 
to a random plan. 
\end{inparaenum}
We propose to support campaigns containing $O(10)$ workflows with 
$O(1000)$ tasks as our initial goal. We will implement these contributions 
into a software prototype designed to execute ecological and biomolecular use 
cases on high performance computing (HPC) resources, using an existing 
workflow execution framework as implemented by RADICAL-EnsembleToolkit 
(EnTK)~\cite{balasubramanian2018harnessing}.

The makespan of a campaign is the time needed to execute all the workflows of 
the campaign, or alternatively, the maximum execution time among all paths 
throughout the campaign~\cite{chirkin2017execution}. Several methods have been 
proposed~\cite{lu2019review} to calculate the makespan of workflows, including 
queuing network~\cite{yao2019throughput,bao2019performance}, and domain specific 
languages~\cite{carothers2017durango,maheshwari2016workflow}. We propose to 
extend the Heterogeneous Earliest Finish Time (HEFT)~\cite{topcuoglu2002performance} 
algorithm to calculate the makespan and schedule of a computational campaign. 
Among the alternatives to HETF, queuing networks is likely to be of limited use 
because they require the user to provide an equivalent queuing network of the 
campaign; domain specific languages approaches would require too much engineering 
effort to convert a workflow representation based on domain specific assumptions, 
e.g. MPI style workflow, or specific languages representation, e.g Swift.

HEFT~\cite{topcuoglu2002performance} is an offline scheduling algorithm which 
calculates the makespan of a workflow on heterogeneous resources. HEFT uses a 
matrix to represent execution time of tasks on resources, assigning tasks to 
the resource that minimizes the finish time of the task. HEFT has complexity 
proportional to the number of dependencies between tasks and the number of 
resources offered. Furthermore, there has been some initial research to extend 
HEFT to resources that provide CPU and GPUs~\cite{shetti2013optimization}. 

Calculating the makespan of a computational campaign depends on workflow 
characteristics, workflow dependencies, campaign dynamicity, and resource 
availability and dynamicity. Task execution time depends on parallelism, 
coordination between tasks, task characteristics~\cite{khoshlessan2017parallel}, 
the framework used to support task execution~\cite{paraskevakos2018task}, and 
resource dynamicity and performance 
variations~\cite{paraskevakos2019workflow, pouchard2019computational}. We propose to utilize 
approximations of workflow execution time that we will experimentally derive.

Based on workflow tasks’ characteristics, resources may need to be configured 
with different execution engines. Compute intensive tasks on HPC resources are 
mainly executed via OpenMP/MPI benefiting from parallelism at scale, while data 
intensive tasks via data parallel frameworks such as Hadoop, Spark~\cite{zaharia2010spark}, 
or Dask~\cite{rocklin2015dask}. Traditionally, HPC resources are designed to 
optimize the execution of MPI tasks, leaving data-parallel frameworks largely 
unsupported. In Ref.~\cite{luckow2016hadoop}, we show how a pilot-based 
middleware~\cite{merzky2019using} can support the efficient and scalable execution 
of data-parallel applications on HPC resources. Such capabilities are necessary 
to support and plan the execution of a data-intensive campaign.

\begin{figure}[t]
	\centering
	\includegraphics[width=.85\textwidth]{CEM_RefArch.pdf}
    \caption{Reference Architecture of a Campaign Manager. Basic 
    subcomponents of Campaign Manager (CM): 1) Makespan Calculation, and 2) Executor. 
    CM communicates decisions to RADICAL-EnTK. CM communicates with HPCs to 
    execute parts of the campaign.\vspace{-2.5ex}}\label{refarch}
\end{figure}

We propose to design a campaign manager (CM) which, given a campaign, an objective, 
and a set of constraints, can derive an execution plan by utilizing the proposed 
makespan HEFT method. Figure~\ref{refarch} shows a reference architecture where 
the CM has two sub-components: 
\begin{inparaenum}[(1)]
\item Makespan Calculator which implements HEFT, and 
\item an Executor which executes the plan.
\end{inparaenum}
Workflow execution will be done through RADICAL-EnTK on HPC resources. If necessary, 
CM will adapt the execution plan by updating the workflows to resource mapping 
decisions. These updates will be based on workflows execution metrics provided 
by EnTK such as tasks execution time, overheads calculation and time to completion. 
These metrics will be aggregated across workflows resulting in campaign-wide 
execution metrics.

To understand the performance of the proposed approach, we will compare it with 
the baseline of performance of a randomly decided execution plan. We propose to 
model a random decision as the mapping of a campaign’s workflows to resources 
based on a uniform distribution between resources. This comparison will validate 
our approach, offering a better understanding of the problem’s requirements for 
further research and development.

Compared to existing solutions, our research will offer engineering, usability 
and possibly performance benefits. The design, as well as the prototype, of the 
campaign manager make minimal assumptions about user-facing applications and  resources. The campaign manager will be designed as a building 
block~\cite{turilli2019middleware} to easily integrate with other workflow 
management frameworks, apart from EnTK. Due to these characteristics, we may 
not be able to directly compare the performance of our campaign manager with 
existing campaign managers. For example, PanDA~\cite{maeno2008panda}, possibly 
the largest campaign manager, is designed to support only the ATLAS 
experiment~\cite{atlas} workflows across hundreds specifically-configured 
grid sites. Similar considerations apply to other workflow managers with 
some campaign management capabilities as DIRAC~\cite{casajus2010dirac}, 
GlideinWMS~\cite{sfiligoi2008glidein} or Pegasus~\cite{deelman2015pegasus}.

Given the size of the problem space and that the proposed work will be performed 
in no more than one year, we propose to focus on the three main contributions described above. In more detail, we will focus on: 
\begin{inparaenum}[(1)]
\item evaluating and deriving the makespan for a campaign as a set of independent 
static O(10) workflows with heterogeneous resource requirements provided by 
ecological and biomolecular sciences use cases on static resources; 
\item offering execution planning capabilities to minimize the makespan of a 
campaign; and 
\item validating our planning capability by executing the workflows of our use 
cases and measuring the accuracy of the estimated campaign runtime and planned 
execution
\end{inparaenum}. In addition, we will explore the requirements to support 
campaigns with dynamic workflows.

To this end, we propose to achieve the following objectives with an estimation 
of the time needed:
\begin{enumerate}
    \item Derive the makespan for a campaign from the ecological sciences and validate based on a random plan. Duration 3 months.
    \item Develop a campaign execution management framework, which will derive and execute a plan based on the results of objective 1. Duration 4 months, overlap with the last month of objective 1.
    \item Integration with scientific workflows and experimentally validate the 
    plan by comparing execution performance with random baseline performance.
    Duration 4 months. The first two months overlap with objective 2.
\end{enumerate}
\bibliographystyle{unsrt}
\bibliography{ext}
\end{document}