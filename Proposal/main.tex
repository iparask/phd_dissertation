\documentclass{article}

%% Language and font encodings
\usepackage[english]{babel}
\usepackage[utf8x]{inputenc}
\usepackage[T1]{fontenc}
\usepackage{float}
\usepackage{listings}
\usepackage{nicefrac}
%% Sets page size and margins
\usepackage[letterpaper,top=3cm,bottom=2cm,left=3cm,right=3cm,marginparwidth=1.75cm]{geometry}

%% Useful packages
\usepackage{amsmath}
\usepackage{amsthm}
\usepackage{graphicx}
\usepackage{color}
\usepackage{xcolor}
\usepackage{cleveref}
\usepackage{booktabs}
\usepackage{multirow}
\usepackage{paralist}
%\definecolor{codegreen}{rgb}{0,0.6,0}
%\definecolor{codegray}{rgb}{0.5,0.5,0.5}
%\definecolor{codepurple}{rgb}{0.58,0,0.82}
%\definecolor{backcolour}{rgb}{0.95,0.95,0.92}

\newif\ifdraft
\drafttrue
\ifdraft
\definecolor{ocolor}{rgb}{1,0,0.4}
\newcommand{\jwave}[1]{ {\reduwave{#1}}}
\newcommand{\jhanote}[1]{ {\textcolor{red} { ***shantenu: #1 }}}
\newcommand{\mtnote}[1]{ {\textcolor{cyan} { ***matteo: #1 }}}
\definecolor{orange}{rgb}{1,.5,0}
\definecolor{dandelion}{cmyk}{0,0.29,0.84,0}
\newcommand{\gpnote}[1]{{\textcolor{green} {***giannis: #1}}}
\newcommand{\note}[1]{ {\textcolor{magenta} { ***Note: #1 }}}
\else
\newcommand{\jwave}[1]{}
\newcommand{\jhanote}[1]{}
\newcommand{\mtnote}[1]{}
\newcommand{\gpnote}[1]{}
\newcommand{\note}[1]{}
\fi

\theoremstyle{definition}
\newtheorem{defn}{Definition}[section]

\lstdefinestyle{mystyle}{
    backgroundcolor=\color{backcolour},   
    commentstyle=\color{codegreen},
    keywordstyle=\color{magenta},
    numberstyle=\tiny\color{codegray},
    stringstyle=\color{codepurple},
    basicstyle=\footnotesize,
    breakatwhitespace=false,         
    breaklines=true,                 
    captionpos=b,                    
    keepspaces=true,                 
    numbers=left,                    
    numbersep=5pt,                  
    showspaces=false,                
    showstringspaces=false,
    showtabs=false,                  
    tabsize=2
}
 
\lstset{style=mystyle}

\title{Autonomic workload manager for executing scientific workflows on High Performance Computing resources}
\author{Ioannis Paraskevakos \\	Electrical and Computer Engineering \\
        Rutgers, The State University of New Jersey}

\begin{document}
\maketitle

\abstract{Many scientific application workflows are becoming larger and need to 
be executed multiple times with different parameters, generating a scientific 
computational campaign. Domains such as molecular dynamics, biological, and 
environmental sciences, define workflows and pipelines that are executed $O(1k)$ 
to $O(10k)$ times. In addition, based on results obtained during runtime the 
application computational requirements may change. Managing computational 
resources to reduce time to completion, maximize utilization, and maximize 
scientific insight, is becoming necessary. In this proposal, we present research 
done to understand:
    \begin{inparaenum}[(i)]
        \item scientific application time to completion characterization,
        \item scalability behavior of scientific workflows based on programming models.
       \end{inparaenum}
We summarize preliminary results from our research for modeling scientific workflows 
in terms of time to completion, and propose an autonomic workload manager to manage 
workflow execution on High Performance resources. We include a work plan to investigate, 
formulate and implement the proposed workload manager and evaluate foreseen challenges 
and risks}


\section{Introduction}
Scientific applications are benefiting from executing multiple workflows, with or without dependencies amongst them, to achieve scientific insight.
Biomolecular sciences, for example, execute workflows simulating physical systems with different initial conditions to enable better drug discovery~\cite{dakka2018concurrent}.
Ecological sciences that use very high resolution (VHR) satellite imagery, require the analysis of TB of data from different calendar years to create time series of ecological changes.
This departs from the approach where the user submits a single workflow and wait for the final results.
Instead the user monitors the execution of several workflows, selects resources, and submits workflows for execution, while coordinating data analysis and workflow adaptation.
This way of execution is called executing a computational campaign.

Computational campaigns enact an execution plan to allow users to achieve a computational objective under given requirements and constraints.
A computational objective is a set of values selected by the user for a set of metrics, e.g., time to completion and throughput, which can be represented as an objective function.
Requirements describe the minimum amount and type of resources needed to execute each workflow of the campaign, while constraints are the conditions that bound the execution, including but not limited to, resource availability, capacity or costs.
A plan describes a sequence of actions that solves the objective function as, for example, selecting, acquiring and configuring resources, and establishing the execution order of workflows on resources.


Planning and enacting the execution of a campaign poses four main challenges: 
\begin{inparaenum}[(i)]
\item evaluating the makespan of a campaign on heterogeneous and dynamic resources;
\item the conditions under which an execution plan performs better compared to a random resource selection;
\item determining a campaign execution plan on available resources that satisfies the given objective function, requirements and constraints of a campaign, while accounting for resource dynamism; and
\item adapting the plan in case of deviation from the objective achievement.
\end{inparaenum}

In this proposal, we propose a campaign manager for executing scientific computational campaigns on high performance computing resources.
This campaign manager, given a campaign, an objective and a set of constraints, will derive and enact upon an execution plan. 
In case, the plan deviates from achieving the objective of the campaign the campaign manager will adjust it.

During the execution of this proposal, we will:
\begin{inparaenum}[(1)]
\item select an algorithm to calculate the makespan of a campaign and determine an execution plan on possibly heterogeneous and dynamic resources;
\item a software system that implements the makespan algorithm, and executes a campaign on resources; and 
\item a method to evaluate the performance of our approach compared to a random plan. 
\end{inparaenum}

The proposal is organized as follows.
Section~\ref{definitions} provides definitions for the used terminology.
Section~\ref{current_research} provides an overview of the research done to support scientific data-intensive workflows on HPC resources, as well as the state of the art on executing scientific campaigns on resources.
Section~\ref{research} presents the research required to design and implement the proposed campaign manager.
Section~\ref{timeline} concludes the proposal by describing the proposal's timeline, significance and impact of the work, and the challenges of this work.

\section{Working Definitions}
\label{definitions}
Terms such as ``execution plan'', ``workflow'', ``computational campaign'', and others are usually overloaded in literature and defined differently across domains. We define the terminology used in this proposal:
\begin{itemize}
    \item \textbf{Computational Campaign} is a set of heterogeneous workflows with or without dependencies amongst them, that need to be executed to achieve a computational objective.
    \item \textbf{Workflow} represents the computational instance implementing an, or part of, application with specific parameter values, number of tasks, task dependencies, and other computational aspects.
    \item \textbf{Makespan} of a campaign is the time needed to execute all the workflows of the campaign, or alternatively, the maximum execution time among all paths throughout the campaign~\cite{chirkin2017execution}
    \item \textbf{Computational Objective} is a set of values selected by the user for a set of metrics which can be represented as an objective function.
    \item \textbf{Requirements} describe the minimum amount and type of resources needed to execute  each workflow of the campaign.
    \item \textbf{Constraints} are the conditions that bound the execution, including but not limited to, resource availability, capacity or costs.
    \item \textbf{Execution plan} is a sequence of actions that solves the objective function as, for example, selecting, acquiring and configuring resources, and establishing the execution order of workflows.
\end{itemize}


\section{Current Research}

TBD.

\subsection{Related Work}
\label{relatedwork}
The SelfLet framework~\cite{bindelli2008building} is an autonomic software system. 
This system provides a set of autonomic components, called SelfLets, that operate 
in order to achieve a goal. Each SelfLet provides a set of services, behaviors 
and policies. A SelfLet can be part of a network which allows it to utilize 
services from other SelfLets to achieve its goal. A SelfLet system has been used 
in a distributed sense to achieve load balanced service requests~\cite{calcavecchia2010emergence}.

The DIOS++ framework~\cite{liu2003dios} offers a rule-based autonomic management 
system for scientific applications. DIOS++ provides abstractions to create sensor 
and actuator which allow runtime monitoring and control, a distributed network to 
connect and manage the sensor/actuators and a distributed engine to execute parts 
of the application based on user defined rules.

Cloud4IoT~\cite{pizzolli2016cloud4iot} is an autonomic system which allows automatic 
deployment and configuration of data-intensive workloads on cloud and edge devices, 
and Internet-of-Things (IoT) application support. Cloud4IoT proposes an architecture 
where a Cloud is used as a central entity of computation, Edge devices are used to 
execute the initial processing steps of a data-intensive application, and IoT 
gateways are used as the gateways for sensors to connect to the system and 
transmit data.

CometCloud~\cite{diazmontes2015cometcloud} is an autonomic software system that 
enables scientific applications on a federation of resources. It has been used 
to enable simulation and data-intensive applications on heterogeneous resources, 
HPC and Cloud, that consumed million of core hours.

Pandey et al.~\cite{pandey2012autonomic} are proposing an autonomic cloud 
environment for executing multi user Electrocardiogram (ECG) data analysis 
workflows. The authors propose a three layer architecture. The ECG analysis 
software is the top layer of the architecture. The second layer contains the a 
scaling manager, a workflow engine, and a workload distribution engine. The cloud 
infrastructure and a authentication mechanism create the third layer of the 
architecture. All three layers are provided as a Service.

\gpnote{Workload management systems..........}


\subsection{Conceptual Model for Task-Parallel framework selections}
\input{conceptual_model.tex}
\subsection{Data Analysis Design Selection}
\gpnote{Write about publication number 2}

\section{Proposed Research}

In our research so far, we discussed a conceptual model for data analytics for 
various use cases on HPC system, and a design comparison and execution time model 
for scientific workflows. In addition, we identified a need for execution of 
scientific workflows with minimum user intervention, independent from users' 
scientific domain. In this section, we motivate and propose an autonomic middleware 
for configuring, monitor and adapting the execution of scientific workflows on 
HPCs.

\subsection{Proposed Topic}

\label{research}
So far, we discussed an extension of the pilot abstraction to support data analytics and task based data intensive applications on HPC resources, a comparison between different task-based data oriented frameworks, and a design comparison for scientific workflows.
In addition, we identified a need for the execution of scientific computational campaigns with minimum user intervention, independent from users' scientific domain.
In this section, we motivate and propose a Campaign Manager (CM) for creating and enacting a campaign execution plan.

\subsection{Motivation}
\label{subsec:motivation}
Campaigns from the biomolecular and earth sciences are diverse in terms of composition, number and size of workflow members, and dynamicity.
Biomolecular science campaigns may be comprised from a small number of workflows with millions of tasks, or thousands of workflows with tens to hundreds tasks~\cite{dakka2018high}. 
Earth sciences campaigns, especially those which use VHR satellite imagery, comprise of workflows with thousands of tasks.
The number of workflows depends on the number of images the user has access to as well as the time they are able to obtain imagery.
These workflows can be static~\cite{paraskevakos2019workflow} or dynamic~\cite{dakka2018high}.

Resource requirements of these campaigns are heterogeneous, in terms of number of resources required and type. 
Biomolecular science software tools support MPI/OpenMP, GPUs, and other accelerators, such as Intel Phi processors~\cite{cheatham2015impact} for executing either simulations or analysis.
In addition, some biomolecular analysis tool require data oriented frameworks, such as PMDA~\cite{fan2019pmda}.
Earth sciences applications, based on imagery, are creating workflows that are executing a series of CPU based preprocessing, and eventually execute a computer vision algorithm on GPUs~\cite{paraskevakos2019workflow}.
As a result, the user needs to manage different types of resources as well as resource requirements for each workflow.

Scientific computational campaign workflows can be executed on a number of HPC resources as long as they meet the workflows' computational requirements.
Despite the fact, users tend to execute all their workflows on one or two systems, under-utilizing their allocations and increasing the makespan of the campaign.
Being able to distribute workflows on different resources will increase the resource utilization of the campaign, but also provide opportunity for optimizing the execution time of the campaign.

Based on the above, there is a need for a system that supports the execution of campaign with heterogeneous resource requirements.
This system should allow users to describe a set of workflows with their computing requirements, the campaign's objective, a set of resources along with the allocations on those resources, and any other information the user deems important, for example workflow importance.
Based on this information, the system should be able to derive and enact an execution plan without any further user intervention.
As a result, users will be able to compile computational campaigns that allow them to execute scientific experiments to sizes prior unfeasible, at least in terms of number of workflows.


\subsection{Proposed Topic}

% ------------------------------------------------------------------------------
% Scientific campaings
Scientific workflows are generally described by a direct acyclic graph, where the nodes are tasks and the edges dependencies.
A subset of this general description of workflows are those which can be represented via the Pipeline, Stage, Task (PST) model~\cite{balasubramanian2018harnessing}.
The PST model describes workflows as a set of pipelines, where each pipeline is a sequence of stages.
Each stage then is a set of tasks that need to be executed.
Concurrency is achieved in the level of pipelines and the level of tasks. 
Based on our use cases, we particularly interested in scientific campaigns comprised by workflows that can be described via the PST model.
Figure~\ref{fig:bio_earth_workflows} shows two example workflows from the biomolecular sciences (Fig.~\ref{fig:bio_workflow}) and the earth sciences (Fig.~\ref{fig:earth_workflow}). 

\begin{figure*}[ht!]
    \centering
    \begin{subfigure}[b]{0.45\textwidth}
        \includegraphics[width=\linewidth]{figures/bio_workflow.pdf}
        \caption{}
        \label{fig:bio_workflow}
    \end{subfigure}%
    ~ 
    \begin{subfigure}[b]{0.45\textwidth}
        \includegraphics[width=\linewidth]{figures/earth_workflow.pdf}
        \caption{}
        \label{fig:earth_workflow}
    \end{subfigure}
    \caption{Biomolecular and Earth Science example workflows. \ref{fig:bio_workflow} Biomolecular workflow based on PST model example~\cite{dakka2018concurrent}; \ref{fig:earth_workflow} Earth science workflow based on PST model example~\cite{paraskevakos2019workflow}}\label{fig:bio_earth_workflows}
\end{figure*}


% ----------------------------------------------------------------------------
% campaign makespan modeling
Achieving a campaign computational objective, such as time to completion or throughput, would require calculating and optimizing the makespan of the campaign.
The makespan of a campaign is the time needed to execute all the workflows of the campaign, or alternatively, the maximum execution time among all paths throughout the campaign~\cite{chirkin2017execution}.
We propose to utilize and extend the Heterogeneous Earliest Finish Time (HEFT) algorithm~\cite{topcuoglu2002performance} algorithm.
HEFT is an offline scheduling algorithm which calculates the makespan of a workflow on heterogeneous resources.
HEFT uses a matrix to represent execution time of tasks on resources, assigning tasks to the resource that minimizes the finish time of the task, and has complexity proportional to the number of dependencies between tasks and the number of resources offered.
Furthermore, there has been some initial research to extend HEFT to resources that provide CPU and GPUs~\cite{shetti2013optimization}, as well as a method to extended HEFT on dynamic resources~\cite{dong2007pfas}.

There are several alternative methods and algorithms to calculate and optimize the makespan of a workflow~\cite{lu2019review}, including queuing networks~\cite{yao2019throughput,bao2019performance}, domain specific languages~\cite{carothers2017durango,maheshwari2016workflow}, and machine learning~\cite{witt2019predictive,pumma2017runtime}.
Queuing networks will be of limited use because they require from the user to provide a queuing network equivalent of the campaign.
A campaign is a set of workflows, and as a result having the user provide a queuing network representation of the campaign adds an additional layer of complexity.
Domain specific languages would require too much engineering effort to convert a workflow representation based on domain specific assumptions, e.g. MPI style workflow, or specific languages representation, e.g Swift, to a PST model representation.
Machine Learning approaches would require model training, validation and testing to produce a model.
In addition, since the execution is done on dynamic resources, the model should be retrained after every workflow execution.

% ----------------------------------------------------------------------------
% Initial Assumptions
Initially, we will assume a static campaign with static workflows, executing on static resources. 
This will allow us to understand the selected algorithm, and find its performance compared to a random plan.
Next, we will relax the assumption of static resources and work with dynamic resources.
We will introduce resource dynamicity as resource availability.
Resources are not available for multiple reasons, including failures and scheduled downtime for maintenance.

HEFT assumes static resources.
Introducing resource dynamicity requires to further extend HEFT to take into account resource availability to decide on which resources it will schedule workflows, as well as update the schedule as resource become unavailable.
In addition, we will try to further relax our assumptions and introduce dynamic campaigns, and dynamic workflows.
% ----------------------------------------------------------------------------
% Campaign Manager definition, requirements, features and capabilities
We propose to design a campaign manager (CM) which, given a campaign, an objective, and a set of constraints, can derive an execution plan by utilizing the proposed makespan HEFT method.
Execution planning for workflows are provided by Pegasus~\cite{deelman2015pegasus}, and ASKALON~\cite{fahringer2005askalon}.
We plan to extend these capability to campaigns.
Figure~\ref{fig:refarch} shows a reference architecture where the CM has two sub-components:
\begin{inparaenum}[(1)]
\item Makespan Calculator which implements HEFT, and
\item an Executor which executes the plan. 
\end{inparaenum}
Workflow execution will be done through an existing workflow management framework (WMF) on HPC resources.
If necessary, CM will adapt the execution plan by updating the workflows to resource mapping decisions. 
These updates will be based on workflows execution metrics provided by the selected WMF such as tasks execution time, overheads calculation and time to completion.
These metrics will be aggregated across workflows resulting in campaign-wide execution metrics.

\begin{figure*}[t]
    \centering
    \includegraphics[width=.95\textwidth]{figures/CEM_RefArch.pdf}
    \caption{Reference Architecture of a Campaign Manager. Basic 
    subcomponents of Campaign Manager (CM): 1) Makespan Calculation, and 2) Executor. 
    CM communicates decisions to RADICAL-EnTK. CM communicates with HPCs to 
    execute parts of the campaign.}\label{fig:refarch}
\end{figure*}

The Makespan Calculator will be responsible for calculating and optimizing the makespan of a campaign based on a set of workflows, a set of resources, and the objective.
Calculating the makespan requires the calculator to have knowledge or derive of the execution time of the workflows that comprise the campaign on the available resources, as well as their availability.
HPC resources are dynamic, in the sense that their performance~\cite{pouchard2019computational} and availability fluctuates.
As a result, the Makespan Calculator requires to take into account performance fluctuations as workflows are being executed.
In addition, the Campaign Manager should be able to verify whether a resource is available and update the plan accordingly.

The Executor sub-component is responsible to execute, and monitor the plan by interfacing with a WMF.
Based on the plan the Makespan calculator decided, the Executor submits workflows to a WMF to execute on the selected resource.
This requires the Executor to work upon multiple workflows concurrently.
In addition, it should monitor workflow execution and resource availability.
An important requirement for the executor is to identify the reason of a failing workflow.
When the failure is because the resource is not available the specific workflow may need to be rescheduled and the plan to be updated.

RADICAL-Ensemble Toolkit~\cite{balasubramanian2018harnessing} (EnTK) is a workflow management framework.
We selected to utilize EnTK because it fits the requirements of the target use case as they are described in \S~\ref{subsec:motivation}
EnTK defines workflows as a set of pipelines, each pipeline is a sequence of stages, and in turn each stage a set of tasks.
EnTK support the execution of a sequence of workflows may either reuse resources or request new ones, based on how the user has programmed the application.
EnTK workflow execution is stateful, provides execution metrics, such as task execution time, and supports workflow execution on multiple HPC resources.
Furthermore, it support a pilot runtime system, RADICAL-Piliot~\cite{merzky2019using}, to execute workflows on HPC resources.
Pilot systems submit job placeholders on resources, and are able to execute tasks on the acquired resources.
The proposed campaign will interface with EnTK to execute workflows based on derived execution plan.


\subsection{Proposed Timelime}
\label{timeline}
Given the size of the problem space and that the proposed work will be performed in no more than a year, we propose to focus on three main contributions:
\begin{inparaenum}[(1)]
\item evaluating and deriving the makespan for a campaign as a set of independent static O(10) workflows with heterogeneous resource requirements provided by ecological and biomolecular sciences use cases on dynamic resources; 
\item offering execution planning capabilities to minimize the makespan of a campaign; and 
\item validating our planning capability by executing the workflows of our use cases and measuring the accuracy of the estimated campaign runtime and planned execution compared to a random plan.
\end{inparaenum}
In addition, we will explore the requirements to support campaigns with dynamic workflows.

To this end, we propose to achieve the following objectives with an estimation of the time needed:
\begin{enumerate}
    \item Develop a campaign manager prototype, which will derive and execute a plan. Duration 3 months
    \item Implement proposed makespan algorithm for executing a campaign from the ecological sciences. Duration 4 months. It overlaps with the last month of phase 1.
    \item Experimentally measure the performance of the selected makespan algorithm for a scientific campaign. Duration 5 months. It overlaps with objective 2 
\end{enumerate}
Figure~\ref{fig:work_plan} shows the Gantt chart of the proposed work.
Subsections~\ref{obj1},~\ref{obj2}, and~\ref{obj3} provide more details as to what will be achieved during each phase of the proposal.
In addition, we allocate three months to accommodate foreseen challenges and risks.
\begin{figure*}[t]
	\centering
	\includegraphics[width=.95\textwidth]{figures/phd_plan.pdf}
	\caption{Planned timeline of proposed research}\label{fig:work_plan}
\end{figure*}

\subsubsection{Phase 1: Design and implementation of a campaign manager}
\label{obj1}

Phase 1 would include design discussions for a prototype of the campaign manager.
Designing a prototype is an iterative process, and it will provided the basic functionality of the campaign manager.
These discussions will result to the requirements of the campaign manager and its API.

There are several methods to design a CM.
PanDA~\cite{maeno2008panda} and glideinWMS~\cite{sfiligoi2008glidein} are utilizing a dedicated server to hold workflows and schedule them to resources.
Balsam~\cite{salim2019balsam} utilizes a database to hold the campaign and workers are pulling workflow tasks for execution.
DIRAC~\cite{casajus2010dirac} creates a set of queues that hold different sized workflow tasks, and use dedicated workers for each queue.
Understanding whether existing design approaches fit the requirements of our use cases is necessary to finalize the CM design.

A prototype will be implemented and will provide campaign execution capabilities.
The prototype will be implemented in python and interface with RADICAL-EnTK as its WMF.
Initially, we will assume that the workflows of the campaign are described based on the EnTK's API.
This will allows us to quickly develop the prototype for the campaign manager.
If there is time in phase 1 we will relax this assumption and try to support different methods of describing workflows based on our use cases.
This will require to translate a workflow from any representation to EnTK's.
Furthermore, the planner component should allow the support multiple makespan algorithms.
Success of this phase will provide an installable python package. 


\subsubsection{Phase 2: Implementation and performance analysis of makespan algorithm}
\label{obj2}
Phase 2 includes implementing the proposed HEFT algorithm in the CM prototype, and characterizing its performance.
Initially, we will implement a random planner, while we extend and implement HEFT to support a campaign.
We will first use a campaign with homogeneous workflows and homogeneous resources to validate our implementation (different mappings of workflows on resources should all have the same makespan as explained in~\S\ref{sec:proposed}). We will then use heterogeneous workflows and static homogeneous resources, eventually moving to static heterogeneous resources and finally to dynamic heterogeneous resources.
This will allow us to characterize the performance of our proposed algorithm, compared to a random plan.

Preliminary experimentation during this phase will provide us with information on whether HEFT can support a campaign execution.
In case it does, we will proceed with phase 3 as soon as possible.
Otherwise, we will research algorithms that can support campaign execution and implement them, following the same approach described above. Even a negative result will be interesting, seeing the lack of current analysis about planning for campaign managers.

\subsubsection{Phase 3: Experimental performance analysis}
\label{obj3}
Phase 3 of the proposed research plan includes an experimental performance analysis of the campaign manager for a set of selected use cases.
We will execute campaigns based on our use cases, with and without our campaign manager, for different campaign sizes.
Based on the gathered data, we will measure metrics such as makespan, resource utilization.
This performance analysis will compare the execution of a computational campaign with and without our CM prototype.
This comparison will provide us with information about whether and when a CM should be used, and about whether the CM performance correlates to the size and characteristics or the campaign's workflow and available resources.

% ---------------------------------------------------------------------------
% Why
\subsection{Significance and impact of work}
Several scientific campaigns require to execute a large number of workflows several times with different input data or initial conditions. 
The required concurrency to minimize the execution time of the campaign is not necessarily constant and may change based on resource availability. 
The campaign manager suggested in this proposal will be the first that offers domain and resource agnostic campaign execution, as well as makespan minimization capabilities. 
This will lead to less time invested by users to make execution decisions about their campaigns. 
These decisions will lead to better resource utilization and, as a result, better domain science. 
The empirical performance analysis derived by this work can be used to derive empirical models initial and eventually formal mathematical models.

This work will have immediate impact on use cases from earth science domains that analyze very high resolution satellite imagery.
These sciences want to analyze imagery from different calendar years and are executing multiple workflows for their analysis.
In addition, new imagery is becoming available in a constant low rate stream.
Utilizing the proposed campaign manager, they will be able to continuously executing workflows that analyze imagery as it becomes available.


% ---------------------------------------------------------------------------
% Challenges
\subsection{Challenges/Risks}

We estimate the proposed work, divided into three major phases, to take 9 months and we allocate 3 months to account for unforeseen circumstances. We would like to keep the committee aware of the following challenges that we see:

\begin{itemize}
	\item Design and Implementation (phase 1) is iterative and special attention needs to be given to the number of iterations against specific objectives, given the timeline.
    \item All experiments performed on HPC systems are subject to variable queue times and may limit the number of experiments performed in phase 2 and 3.
	\item Although the campaign manager will be well tested (80--90\% of the code base will be covered by unit tests) and less susceptible to major changes, RADICAL-EnTK's runtime system, RADICAL-Pilot is known to be less stable and is susceptible to changes as it serves multiple projects.
\end{itemize}

\bibliographystyle{plain}
\bibliography{proposal}
\end{document}